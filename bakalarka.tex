%!TEX program = xelatex
\documentclass[a4paper,12pt,openany,twoside]{book} % twoside
\usepackage[inner=3.5cm,outer=2.5cm, top=2.5cm]{geometry} %showframe

\usepackage{url}
\usepackage{fontspec}
\usepackage{lmodern}
\usepackage{csquotes}
\usepackage{graphicx}
\usepackage{caption}
\usepackage{subcaption}
\usepackage[rgb]{xcolor}
\graphicspath{ {assets/} }
\usepackage{xunicode}
\usepackage{xltxtra}
\usepackage{chngcntr}
\usepackage[czech]{babel}
\usepackage[pagestyles,medium]{titlesec}
\usepackage{setspace}
\usepackage{emptypage}
\usepackage{xpatch}
\usepackage{caption}
\usepackage{subcaption}
\usepackage{ragged2e}
\usepackage{pbox}
\usepackage{rotating}
\usepackage{float}
\usepackage{afterpage}
\usepackage{tipa}
\usepackage{array}
\usepackage{wrapfig}
\usepackage{multirow}
\usepackage{tabularx}
\usepackage{booktabs}
\newcolumntype{C}{>{\centering\arraybackslash}X}
\usepackage{lscape}
\usepackage{bashful}

\usepackage[hidelinks]{hyperref}

\usepackage{mdframed}
\usepackage{framed}
\def\tightlist{}

\xpatchcmd{\part}{\thispagestyle{plain}}{\thispagestyle{empty}}{}{}


\newcommand\exmp{\textsf}

\counterwithout{figure}{chapter}
\counterwithout{footnote}{chapter}


\newpagestyle{sensible}{
	\headrule\sethead{}{}{\MakeUppercase{\chaptertitle}}
	\setfoot{}{\thepage}{}
}

\setlength\emergencystretch{1em}
\setlength\headheight{14pt}
\setstretch{1.1}
\setcounter{secnumdepth}{4}

\widowpenalty10000
\clubpenalty10000

\hyphenation{выпа-дений}

\setmainfont[Ligatures=TeX]{CMU Serif Roman}

\usepackage[backend=biber, style=iso-authoryear, sortlocale=cs\_CZ, uniquelist=false, autocite=footnote, maxcitenames=2, maxbibnames=99, minnames=1, urldate=long, spacecolon=false,bibencoding=UTF8
]{biblatex}
\let\oldmultinamedelim\multinamedelim
\let\oldfinalnamedelim\finalnamedelim
\renewcommand*{\multinamedelim}{~a~}
\renewcommand*{\finalnamedelim}{~a~}
\renewcommand*{\nameyeardelim}{~}
\AtBeginBibliography{%
  \renewcommand*{\multinamedelim}{~--\space}%
  \renewcommand*{\finalnamedelim}{~--\space}%
}
\DefineBibliographyStrings{czech}{%
  mathesis = {Bakalářská diplomová práce},
  editors = {eds.}
}

\addbibresource{bibliography.bib}

\newcommand{\sign}[1]{%      
  \begin{tabular}[t]{@{}r@{}}
  \makebox[2.5in]{\dotfill}\\
  \strut#1\strut
  \end{tabular}%
}


\begin{document}
	\clearpage
	\pagenumbering{gobble}


		\begin{center}
			{\Large\uppercase{Masarykova univerzita}}

			\vspace{2em}

			{\Large Filozofická fakulta}

			\vspace{20em}

			{\LARGE Bakalářská práce}

			\vfill
			    {\LARGE 2021}
			    \hfill
            	{\LARGE Kryštof Davídek}
		\end{center}

		\begin{center}
			{\Large\uppercase{Masarykova univerzita}}

			\vspace{2em}

			{\Large Filozofická fakulta}

			\vspace{5em}

			{\Large\bf Katedra informačních studií a knihovnictví}

			\vspace{2em}

			{\Large Informační studia a knihovnictví}

			\vspace{11em}

			{\large Kryštof Davídek }
			
			\vspace{3em}
			
			{\LARGE\bf Návrh a implementace online kurzu pro výuku datové analytiky}

			\vspace{1.5em}

			{\Large Bakalářská práce}

			\vfill
			\vspace{3em}
			{\large Vedoucí práce: RNDr. Michal Černý}
			\vspace{1em}
			
			{\large 2021}
		\end{center}
		
\newpage

\par
\par\vspace*{\fill}
	\pagestyle{plain}
\pagenumbering{roman}
\begin{flushright}
	Prohlašuji, že jsem bakalářskou diplomovou práci vypracoval samostatně s~využitím uvedených pramenů a~literatury.

	\vspace{3em}

	    \makebox[2.5in][r]{\dotfill}
	    
	    Kryštof Davídek

	    \par

\end{flushright}
\clearpage

\par
\par\vspace*{\fill}

Na tomto místě bych rád poděkoval...

\clearpage

\section*{Bibliografický záznam}

DAVÍDEK, Kryštof. \textit{Návrh a implementace online kurzu pro výuku datové analytiky} [online]. Brno. Dostupné z: https://is.muni.cz/th/xzmyx/. Bakalářská práce. Masarykova univerzita, Filozofická fakulta. Vedoucí práce Michal ČERNÝ.

\section*{Anotace}

Předkládaná bakalářská práce si klade za cíl navrhnout a implementovat online vzdělávací kurz, který se zaměřuje na výuku základů datové analytiky. Práce se dělí do dvou hlavních částí – teoretické a praktické. V první částí je představen koncept datové analytiky v kontextu datové a digitální gramotnosti a popsán kompetenční rámec, kterým lze tuto oblast zarámovat. Druhou složkou teoretické části je přehledová studie existujících online kurzů zaměřených na témata související s datovou analytikou. Jejím cílem je zmapovat aktuální stav online kurzů vzniklých primárně v zahraničním prostředí a prostřednictvím předem jasně definovaných parametrů tyto kurzy klasifikovat. Praktická část se následně skládá z popisu vlastního e-learningového řešení z hlediska konceptuálního a obsahového. Implementace kurzu probíhala za spolupráce s firmou Digiskills a pro zpracování online kurzu byly využity interní firemní webové šablony.

\section*{Klíčová slova}
 
Datová analytika, datová gramotnost, e-learning, implementace online kurz, přehledová studie.

\clearpage

\section*{Abstract}

The aim of this bachelor thesis is to design and implement an educational online course focused on teaching the basics of data analytics. The thesis is divided into two main parts – theoretical and practical. The first part introduces the concept of data analytics in the context of data and digital literacy and describes the competence framework by which this area can be framed. The second section of the theoretical part is an overview study of existing online courses focused on topics related to data analytics. It aims to map the current state of online courses created primarily in a foreign environment and to classify these courses using clearly defined parameters. The practical part then consists of a conceptual and content description of the e-learning solution. The course was implemented in cooperation with Digiskills and internal company web templates were used to develop the online course.

\section*{Keywords}

Data analytics, data literacy, e-learning, implementation of an online course, overview study.

\clearpage

\tableofcontents

\cleardoublepage
\pagenumbering{gobble}

% \newgeometry{top=2.5cm}

\pagenumbering{arabic}
\hypertarget{uxfavod}{%
\chapter*{Úvod}\label{uvod}\addcontentsline{toc}{chapter}{Úvod}}

Derivace v~užším smyslu představuje v~českém jazyce, jakožto v~jazyce
s~vysoce rozvinutou flexí, nejčastější způsob utváření nových pojmenování.
Její podstata je nejčastěji založena na přidávání slovotvorného morfému
před (prefix) anebo za (sufix) slovní tvar. Rodilí mluvčí podvědomě
chápou sémantiku jednotlivých morfémů, a~proto si jsou schopni nejen
vyložit významy slov, které v~dané podobě nikdy neslyšeli, ale taktéž
takové novotvary vytvářet. U~cizinců učících se češtinu si je však
potřeba sémantiku těchto odvozovacích prostředků nejprve vědomě osvojit.
V~současné době neexistuje nástroj, který by byl určen pro účely
osvojování derivačních prostředků češtiny.

Cílem této práce je navrhnout a~implementovat elektronický slovník
s~definicemi založenými na derivačních rysech slovotvorně motivovaných
slov. Výsledná aplikace provádí pro zadaný vstup částečnou slovotvornou
analýzu, na základě které zadanému slovu přiřazuje definici vycházející
z~jeho struktury (potažmo strukturního významu).

Teoretickým východiskem pro vývoj aplikace je onomaziologická teorie
slovotvorby představovaná Milošem Dokulilem. Aplikace pracuje s~volně
přístupnými daty derivační sítě DeriNet, jež rovněž vychází z~Dokulilové
teorie. Za využití moderních hybridních technologií pak tato data
zpracovává formou mobilní aplikace.

Práce se skládá ze dvou částí -- teoretické a~praktické. V~teoretické
části jsou nastíněny synchronní přístupy k~české slovotvorbě a~současně
je zde hlouběji popsaná onomaziologická teorie slovotvorby. Dále jsou
zde představeny již existující softwarové nástroje, které s~českou
slovotvorbou pracují.

Praktická část se soustřeďuje na popis výsledného nástroje -- nejprve je
představen proces vytváření slovotvorných definic (spolu s~popisem
zpracovaných slovotvorných sufixů) a~posléze je nástroj popsán
z~hlediska jeho návrhu a~implementace.

\part{Teoretická část}

\hypertarget{datovuxe1-analytika}{%
\chapter{Datová analytika}\label{datovuxe1-analytika}}

Ústředním tématem této práce je fenomén datové analytiky, jejímž význačným rysem je vágnost přesného obsahu, protože se na ni v~odlišných prostředích a~kontextech nahlíží různým pohledem dle měnících se potřeb. Abychom byli schopni provést systematickou přehledovou studii online kurzů zabývajících se touto problematikou a~zároveň mohli při návrhu vlastního e-learningového řešení vycházet z~určitého kompetenčního rámce, se v~této kapitole pokusíme tuto disciplínu zarámovat do konceptů datové a~digitální gramotnosti. Ve druhé části této sekce se pak zaměříme na konkrétní kompetence, jež jsou součástí datové analytiky, a~z~nichž budeme v~dalších částech vycházet.

\hypertarget{datovuxe1-gramotnost}{%
\section{Datová gramotnost}\label{datovuxe1-gramotnost}}

Při práci s~konceptem datové gramotnosti se můžeme setkat s~různými pojetími, které se buď vzájemně překrývají, zastupují či doplňují. Dle některých autorů lze datovou gramotnost zahrnout pod gramotnost informační, protože právě ta může ve své obecné definici zahrnovat i~práci s~daty s~jakožto speciálním druhem informace~\parencite[126]{calzada13}. To znamená, že lze informační gramotnost uvažovat jako prerekvizitu k~datové gramotností, jelikož práce s~daty vyžaduje kritický pohled na informace -- jejich správně čtení, hodnocení a~interpretaci.

Dle uznávaného standardu instituce \emph{ACRL}\footnote{The Association of College and Research Libraries – http://www.ala.org/acrl/} je informační gramotnost definována jako souhrn schopností a~dovedností jedince, které slouží k~identifikaci své informační potřeby a~jejímu následnému uspokojení protřednictvím lokalizace, evaluace (vyhodnocení) a~efektivního využití dané informace~\parencite[2]{acrl14}. Podobným způsobem definuje informační gramotnost i~TDKIV\footnote{https://tdkiv.nkp.cz/}: \uv{Schopnost jedince identifikovat informační potřebu, vyhledat informace, zhodnotit je, zpracovat a~efektivně využít}~\parencite{tdkiv03}.

Bez těchto složek si lze pak těžko představit datovou gramotnost tak, jak je představena ve svém nejobecnějším pojetí:

\begin{itemize}
\tightlist
\item
  přístup k~datům a~jejich sběr;
\item
  posouzení kvality dat;
\item
  manipulace s~daty a~jejich zpracování;
\item
  sumarizace a~zhodnocení dat;
\item
  prezentace dat~\parencite[8]{schield05}.
\end{itemize}

Další autoři pracují s~významem datové gramotnosti ve spojitosti se statistickou gramotností. Pod touto gramotností si lze představit dovednosti spojené se čtením, interpretací a~tvorbou statistických dat jako jsou grafy, tabulky a~číselná data (obecněji řečeno jde o~základní znalost deskriptivní statistiky)~\parencite[8]{schield05}. Nicméně my se budeme v~této práci držet názorového proudu, který tyto specifické dovednosti často techničtější povahy chápe jako podsoučást datové gramotnosti~\parencite[125]{calzada13} .

Prováznost informační a~datové gramotnosti můžeme dále najít ve standardu informační gramotnosti \emph{The seven pillars of information literacy}, který inherentně předpokládá, že právě data jsou s~informacemi vzájemně spjaty, a~proto lze s~nimi pracovat na principiálně podobné, ne-li stejné úrovni~\parencite[126]{calzada13}. Tento standard nabízí sedm pilířů (oblastí), které je nutné v~rámci práce s~informacemi/daty provádět~\parencite{sconul11}. Předchozí výčet dovedností a~schopností bychom tak mohli modifikovat následovným způsobem:

\begin{itemize}
\tightlist
\item
  identifikace datové potřeby;
\item
  zhodnocení aktuálních znalostí a~případné zjištění nedostatků;
\item
  vytvoření strategie pro vyhledání potřebných dat;
\item
  přístup k~datům a~jejich sběr;
\item
  evaluace kvality získaných dat;
\item
  etická manipulace s~daty spolu s~jejich zpracováním;
\item
  prezentace výsledků získaných ze zpracovaných dat.
\end{itemize}

V~literatuře se lze dále setkat s~konceptem \emph{Data information literacy} (DIL), který sice úzce souvisí s~informační (resp. datovou) gramotností, nicméně se s~ním pracuje primárně v~rámci zpracování vědeckých a~vzdělávacích dat~\parencite{jeffryes13}. Jde tedy především o~kompetence spojené se správou výzkumných dat, nicméně i~v~této oblasti lze najít přesahy do některých podoblastí popsaných výše.

Například s~problematikou vizualizace dat lze pracovat jak v~DIL, tak i~na úrovní ostatních kompetenčních rámců, jimž může být třeba již zmíněný standard \emph{Information literacy competency standards for higher education}. Resp. jednotlivé moduly tohoto standardu v~sobě implicitně obsahují témata související s~vizualizací dat -- najít je můžeme konkrétně ve druhém (posouzení), třetím (evaluace) a~primárně čtvrtém (využití) modulu. Důležitým aspektem je pak pasivní přítomnost tohoto tématu napříč kompetencemi, protože ku příkladu ve druhém modulu se předpokládá, že právě ten jedinec, který se dokáže vyznat a~orientovat ve vizuálně laděných informací, může splňovat širší kompetenci týkající se správného zhodnocení informací~\parencite{womack14}. Věříme proto, že jsme schopni naprostou většinu témat, ze kterých sestává oblast datové analytiky (např. již zmíněná vizualizace dat) hledat právě v~těchto standardech, a~je z~nich tedy možné v~obecnějším pojetí vycházet.

V~odborném prostředí si také dále můžeme všimnout tendenci k~edukaci datové gramotnosti prostřednictvím výuky, která u~studentů podporuje rozhodování založených na datech (\emph{data-driven decisions})~\parencite{mandinach13}. Pěkným příkladem může být propojení tohoto přístupu s~konkrétními problémy pocházející z~reálného světa, kde mají studenti všechny principy související s~\emph{data-driven decisions} kombinovat s~tzv. \emph{problem solving} technikami. A~to vše ve spojitosti se zpracováním tabulkových dat v~určitém tabulkovém procesoru~\parencite{slayter17}.

\hypertarget{digituxe1lnuxed-gramotnost}{%
\section{Digitální gramotnost}\label{digituxe1lnuxed-gramotnost}}

Druhým neméně důležitým druhem gramotnosti, který s~našim tématem úzce souvisí, je gramotnost digitální. Tento koncept je dobře popsán ve dvou stěžejních standardech/certifikátech -- jsou jimi evropský rámec \emph{DIGCOMP 2.0}\footnote{https://ec.europa.eu/jrc/en/digcomp/digital-competence-framework} a~mezinárodně standardizované sylaby konceptu \emph{ECDL / ICDL}\footnote{https://icdleurope.org/} (dříve \emph{European Computer Driving Licence}, dnes již přejmenován na \emph{European certification of Digital Literacy}).

Do jisté míry zjednodušení se dá říci, že standard \emph{DIGCOMP 2.0} nabízí potřebný výčet digitální kompetencí spolu s~konkrétními možnostmi realizace\footnote{Jedna se o~pět na sebe navazujících modulů – informační a~datová gramotnost, komunikace a~spolupráce, tvorba digitálního obsahu, bezpečnost, schopnost řešení problémů~\parencite{digicomp17}} a~tyto kompetence se snaží uvést v~univerzální vzdělávací rámec. Na ten pak částečně navazuje koncept \emph{ECDL / ICDL}, jenž v~sobě mimo jiné obsahuje samostatný program \emph{ECDL DIGCOMP}, prostřednictvím kterého lze tyto kompetence ověřit platným certifikátem~\parencite{chabera21}.

V~kontextu obou standardů si můžeme všimnout, že se problematika informační a~datové gramotnosti (a~jejich dalších podkomponent) objevuje na větším množství míst. To, co je ale pro naši charakterizaci datové analytiky významnější, jsou certifikáty \emph{ECDL Advanced}\footnote{https://www.ecdl.cz/cert\_ecdl\_advanced.php}, které obsahují vzdělávací sylaby znalostí a~dovedností potřebných pro získání daného certifikátu. Obsah těchto sylabů je totiž určen pro digitálně kvalifikovanou veřejnost a~pokrývá profesionální uživatelské znalosti a~dovednosti v~oblasti kancelářských aplikací~\parencite{ecdl17}. Obsah tohoto modulu se v~další podkapitole pokusíme zasadit do kontextu požadavků na oblast datové analytiky.

\hypertarget{kompetence-datovuxe9-analytiky}{%
\section{Kompetence datové analytiky}\label{kompetence-datovuxe9-analytiky}}

\hypertarget{pux159uxedbuznuxe9-discipluxedny}{%
\subsection{Příbuzné disciplíny}\label{pux159uxedbuznuxe9-discipluxedny}}

Jedním z~přístupů při hledání konkrétních kompetencí datové analytiky může být porovnání této disciplíny s~oblastmi příbuznými, jako jsou primárně data science a~business analytics\footnote{U těchto termínů budeme v~textu používat jejich anglické varianty, protože se tak používají i~v~rámci českého prostředí. Na druhou stranu u~datové analytiky namísto data analytics volíme českou variantu, protože se náš kurz vztahuje k~českému prostředí, kde chceme tuto oblast popularizovat.}. V~odborných pracích se k~hledání podobností a~odlišností napříč těmito disciplínami často využívá metoda obsahové analýzy, prostřednictvím které lze tyto vztahy zachytit a~popsat. My se zde v~následujících řádcích pokusíme ve stručnosti sumarizovat některé závěry, které z~těchto studií vychází a~které považujeme za relevantní pro naše téma.

První studie se zabývala porovnáním datové analytiky s~doménou data science v~kontextu nabídky univerzitních kurzů. Práce tedy zkoumá obsahy vybraných studijních programů a~snažila se je charakterizovat prostřednictvím popisu jednotlivých dovedností/schopností. Z~textu vyplývá, že hlavní podobnost těchto dvou oborů je v~zakomponování témat vizualizace dat a~práce s~numerickými a~matematickými operacemi. Zajímavostí je, že stejně tak v~obsahu kurzů chybí u~obou oblastí akcentování tématu datové etiky či správy velkých dat~\parencite[109]{aasheim2015}.

Co se týká rozdílných kompetencí, tak hlavní odlišností je, že v~rámci data science se od studentů počítá s~výraznějším vhledem do problematiky lineární algebry (spolu s~diskrétní matematikou) a~statistického programování. U~datové analytiky se naopak prioritizuje používání modelových příkladů k~osvojení aplikací analytických technik a~nástrojů (data science pracuje s~spíše s~tvorbou algoritmů, na nichž jsou tyto nástroje postaveny). Druhým výrazným rozdílem je složka týkající se vizualizací, kterou, jak bylo řečeno výše, obsahují obě disciplíny. Nicméně právě datová analytika vnímá vizualizace jako efektivní prostředek ke komunikaci daných informací, narozdíl od data science, jež s~vizualizacemi pracuje spíše jako s~reprezentaci určitých dat~\parencite[110]{aasheim2015}.

Druhý odborný článek zkoumá rozsah dovedností pro vykonávaní těchto disciplín\footnote{Studie zpracovává mimo dvě zmíněné oblasti ještě domény týkající se business analytics a~business intelligence analytics – my ale tyto dvě oblasti sloučíme, protože se jedná o~obsahově obdobné disciplíny, jejichž vydělení není pro charakterizaci datové analytiky až tak relevantní.} na základě analýzy online inzercí pracovních pozic a~jejich požadavků vztahujících se k~daným oblastem. Výsledkem je i~zde porovnání příbuzných a~rozdílných kompetencí a~je velmi zajímavé, že závěry lze bez velkých obtíží vztáhnout k~výsledkům práce popsané výše. Není tedy velkým překvapením, že i~v~této studii vyšel za hlavní rozdíl mezi datovou analýzou a~data science důraz na technickou stránku věci (programování, práce se statistkou atd.), který je primárně součástí data science. Na druhou stranu zde byl u~obou disciplín popsán požadavek na kompetence spojené s~rozhodováním postaveném na datovém základě -- tzn. dovednosti týkající se tvorby efektivní reportu spolu s~analytických myšlením~\parencite[248]{verma19}.

V~rámci porovnání datové analytiky s~business analytics můžeme vidět větší množství vzájemných přesahů v~oblasti měkkých dovedností (komunikace, týmová spolupráce, organizační dovedností atd.), protože doména business analytics je primárně netechnickou disciplínou, u~níž je zapotřebí minimální vhled do problematiky statistických nástrojů~\parencite[249]{verma19}.

Datovou analytiku lze tedy zařadit mezi obě zaměření, protože její kompetence jsou jak povahy technické ve smyslu aplikace základních analytických nástrojů, tak i~interpretační, jež se týká komunikace vizuálně zpracovaných dat a~následné tvorbě informovaných rozhodnutí.

\hypertarget{kompetenux10dnuxed-ruxe1mec}{%
\subsection{Kompetenční rámec}\label{kompetenux10dnuxed-ruxe1mec}}

Spolu s~informacemi získaných z~minulé podkapitoly k~vytyčení praktických kompetencí využijeme již zmíněný certifikát \emph{ECDL Advanced}, a~to konkrétně \emph{ECDL / ICDL Data Analytics SYLABUS 1.0 (AM8)}, který je jedním z~šesti dostupných modulů tohoto vzdělávacího sylabu. Dle oficiálního znění by měl být úspěšný absolvent schopen: \uv{připravit a~zpracovat libovolná data a~ovládat základy datové a~statistické analýzy a~vizualizaci dat}~\parencite{ecdl17}. Tento cíl je dále rozvrstven do jednotlivých kategorií (např. \emph{příprava datového zdroje}), ze kterých jsou dále vytvořeny oblasti znalostí (např. \emph{import, přizpůsobení importu}) a~následně výčet potřebných konkrétních dovedností (např. \emph{ověřovat, zda data patří do sledované datové sady, pomocí vyhledávacích}).

I~když je tento modul zpracován vyčerpávajícím způsobem (koresponduje s~většinou dovedností popsaných v~předcházejících článcích), tak pro popis kompetencí datové analytiky zcela nepostačuje, protože v~sobě neobsahuje takové náležitosti jako jsou právě interpretace vizualizovaných dat a~hlavně chápání potřeby konat rozhodování na datově podloženém základu. Ostatně právě tyto kompetence jsou podsoučástí nadřazené informační gramotnosti, protože např. bez řádné identifikace informační potřeby můžeme jen těžko nacházet/vytvářet novou informaci, i~kdybychom měli sebevětší datový soubor.

\hypertarget{pux159ehledovuxe1-studie-online-kurzux16f}{%
\chapter{Přehledová studie online kurzů}\label{pux159ehledovuxe1-studie-online-kurzux16f}}

\hypertarget{charakteristika-pux159ehledovuxe9-studie}{%
\section{Charakteristika přehledové studie}\label{charakteristika-pux159ehledovuxe9-studie}}

V~této kapitole teoretické části se budeme zabývat již existujícími online kurzy, které se určitým způsobem soustředí na problematiku datové analytiky či alespoň zpracovávají některou z~jejích vybraných kompetencí. Našim cílem je v~této sekci systematicky kategorizovat významná e-learningová řešení a~na základě této klasifikace vytvořit určitou formu přehledové studie\footnote{Je si však zapotřebí uvědomit, že svojí povahou je tato bakalářská práce nejblíže aplikační kvalifikační práci, jejímž cílem bylo vytvořit funkční online kurz, z~tohoto důvodu nelze považovat tuto přehledovou studii za plnohodnotnou výzkumnou studii, tak jak je uváděna v~odborné literatuře~\parencite{mares2013}.}. Každý zpracovaný online (vybírali jsme pouze takové kurzy, s~jejichž obsahem se dá alespoň částečně bezplatně pracovat, a~mají tedy nějakou podobu otevřeného vstupu) kurz sestává ze tří oddělených ale vzájemně se doplňujících částí.

\hypertarget{anotace-kurzu-s-odux16fvodnux11bnuxedm-vuxfdbux11bru}{%
\subsection{Anotace kurzu s~odůvodněním výběru}\label{anotace-kurzu-s-odux16fvodnux11bnuxedm-vuxfdbux11bru}}

V~rámci anotace popisujeme základní cíle kurzu, ve stručnosti charakterizujeme jeho obsahovou stránku a~nastiňujeme strukturu, z~níž se daný kurz skládá. Dále se snažíme zdůvodnit, proč jsme daný kurz vybrali, a~pokusíme se jej začlenit do některého z~konceptů informační či datové gramotností, případně do dalších kompetenčních rámců.

\hypertarget{tabulka-s-parametry}{%
\subsection{Tabulka s~parametry}\label{tabulka-s-parametry}}

Podstatnou částí analýzy u~každého z~online kurzů je tabulka s~předem definovanými parametry, která systematicky uvádí základní informace a~přehledným způsobem znázorňuje charakteristiku daného kurzu. Konkrétní tabulku s~parametry k~určitého kurzu vždy přikládáme do seznamu příloh a~prostřednictvím odkazu propojujeme s~anotací. Níže uvádíme seznam vybraných parametrů:

\begin{itemize}
\tightlist
\item
  název;
\item
  použitá platforma;
\item
  datum vzniku;
\item
  specifikace obsahu kurzu (jaká témata kurz reflektuje);
\item
  časová náročnost;
\item
  způsob práce s~hodnocením a~úkoly;
\item
  rozsah;
\item
  forma ukončení;
\item
  začlenění do kompetenčního rámce;
\item
  použité technologie.
\end{itemize}

\hypertarget{popis-integrovanuxfdch-prvkux16f}{%
\subsection{Popis integrovaných prvků}\label{popis-integrovanuxfdch-prvkux16f}}

Poslední složkou přehledu je popis prvků, které jsme se rozhodli integrovat do našeho praktického řešení. V~této části se nevážeme pouze na jednotlivá témata či použité nástroje, ale všímáme si také zvolených edukačních prostředků. Může jít tedy například o~způsoby průběžného ověřování znalosti a~dovedností, zvolené aktivizační metody za účelem větší interaktivity nebo možnosti získávání zpětné vazby na jednotlivé obsahové moduly. Tyto prvky nám dále poslouží k~návrhu zbývajících částí online kurzu a~případně k~redesignu stávajících komponent.

\hypertarget{online-kurzy}{%
\section{Online kurzy}\label{online-kurzy}}

\hypertarget{introduction-to-data-analytics-for-business}{%
\subsection{Introduction to Data Analytics for Business}\label{introduction-to-data-analytics-for-business}}

\hypertarget{anotace}{%
\subsubsection{Anotace}\label{anotace}}

Kurz \emph{Introduction to Data Analytics for Business}\footnote{Systematickou analýzu kurzu uvádíme v~tabulce s~parametry viz Tabulka \ref{tab1}.} na platformě Coursera je součástí širší specializace pod názvem \emph{Advanced Business Analytics Specialization} a~je zprostředkován pod záštitou University of Colorado Boulder. Tento kurz je úvodní částí zmíněné specializace a~jeho cílem je popsat základní principy datové analytiky v~podnikovém prostředí. Obsah kurzu se velkou mírou soustředí na integraci datové analytiky do procesů určité organizace a~na konkrétních příkladech ukazuje, proč je důležité tuto oblast reflektovat na individuální úrovni ve smyslu zavádění analytických technik do vlastí pracovních mikroprocesů, a~tím zvyšovat celo-organizační efektivitu. Kurz každopádně nezůstává pouze u~teoretických základů, ale v~pozdějších částech pracuje s~využitím konkrétních nástrojů a~představuje základní analytické techniky na úrovni relačních databází a~dotazovacího jazyku SQL~\parencite{course1}.

Struktura kurzu je tvořena čtyřmi na sebe navazujícími sekcemi:

\begin{itemize}
\tightlist
\item
  \emph{Data and Analysis in the Real World};
\item
  \emph{Analytical Tools};
\item
  \emph{Data Extraction Using SQL};
\item
  \emph{Real World Analytical Organizations}.
\end{itemize}

Tento kurz jsme do našeho výběru zahrnuli jednak proto, že se jedná o~jeden z~nejlépe hodnocených kurzů na platformě Coursera zabývající se našim tématem, a~jednak z~toho důvodu, že se nezaměřuje pouze na představení konkrétního analytického nástroje, ale snaží se reflektovat motivaci, která je příčinou pro využívání datové analytiky v~podnikovém prostředí.

\hypertarget{integrovanuxe9-prvky}{%
\subsubsection{Integrované prvky}\label{integrovanuxe9-prvky}}

Za hodnotnou část tohoto kurzu vnímáme primárně teoretičtější téma využití datové analytiky v~podnikovém prostředí -- konkrétně jde tedy o~první a~čtvrtý modul. Autoři kurzu totiž přehledným způsobem vysvětlují datovou analytiku jakožto nutnou součást firemních procesů, které začínají u~samotných jednotlivců a~časem přesahují do dalších pater dané organizace. Tuto část pokládáme ze velmi důležitou a~pokusíme se ji tak začlenit do úvodních částí našeho vlastního kurzu.

Druhou složkou, jež bychom rádi nějakým způsobem integrovali, je koncept jednoho komplexního úkolů, na němž studenti po celou dobu studia pracují. V~tomto případě šlo sice o~úkol, který se nacházel až v~posledním modulu, nicméně sestával ze všech předcházejících dovedností a~pracoval s~reálným podnikovým modelem. Rádi bychom tedy i~vlastní studenty pobídli k~práci na jednom větším projektu, který se skládá ze vzájemně na sebe navazujících části a~který je co nejvíce přibližuje k~reálnému prostředí určité organizace.

\hypertarget{intro-to-data-analysis}{%
\subsection{Intro to Data Analysis}\label{intro-to-data-analysis}}

\hypertarget{anotace-1}{%
\subsubsection{Anotace}\label{anotace-1}}

Druhý analyzovaný kurz je součástí většího balíku kurzů na online platformě Udacity, které společně tvoří takzvaný \emph{Data Analyst Nanodegree}. Jedná se o~ucelený celek vzdělávacích kurzů, které si kladou za cíl vzdělat studenta na požadovanou úroveň v~dané oblasti pro účely větší konkurenceschopnosti na trhu práce. Výhodou toho programu je nabytí praktické zkušenosti s~konkrétními analytickými nástroji a~technikami, nicméně to vše za cenu poměrně vysokého finančního poplatku.

My se každopádně zaměřujeme na kurz \emph{Intro to Data Analysis}\footnote{Systematickou analýzu kurzu uvádíme v~tabulce s~parametry viz Tabulka \ref{tab2}.}, který je bezplatným úvodním seznámením s~tématem datové analytiky v~rámci celé této specializace. I~přesto je však tento kurz zaměřen převážně prakticky, protože se sám dělí ještě do tří vzájemně navazujících lekcí (spolu se závěrečným úkolem), z~nichž se právě dvě zabývají poměrně pokročilými nástroji (jde o~programovací jazyk Python a~jeho dvě knihovny používané pro statistické účely Numpy a~Pandas).

Pro naše účely je však důležitá primárně první lekce, jež se zabývá kontextualizací datové analytiky jakožto samostatné disciplíny a~která přehledným způsobem demonstruje všechny potřebné složky, které jsou součástí procesu datové analytiky. V~pojetí tohoto prakticky zaměřeného kurzu jde tedy o: \emph{question} (jaký problém chceme analýzou řešit), \emph{wrangle} (přístup k~datům a~jejich čištění), \emph{explore} (vlastní explorativní analýza na základě očištěných dat), \emph{draw conclusions} (získání nové informace) a~\emph{communicate} (komunikace výsledné informace prostřednictvím vizualizace)~\parencite{course2}.

\hypertarget{integrovanuxe9-prvky-1}{%
\subsubsection{Integrované prvky}\label{integrovanuxe9-prvky-1}}

I~přes to, že svým praktickým pojetím tento kurz značně přesahuje potřeby, které jsou kladeny na naše vlastní řešení, lze využít použitého konceptu procesu datové analytiky. Autoři kurzu tímto způsobem efektivně vydělili a~zpracovali jednotlivé kroky, které vedou k~úspěšnému provedení datové analytiky -- my tak můžeme pro naši praktickou část podobně použít obsah dokumentů \emph{ECDL / ICDL Data Analytics SYLABUS 1.0 (AM8)} (viz kapitola \ref{kompetenux10dnuxed-ruxe1mec}) a~přetvořit jej do podobně strukturované podoby, jako byla použita v~tomto kurzu.

Druhý významný přínos je pro nás ve srozumitelném představení základních příkladů, kdy můžeme datovou analytiku využít na dennodenní bází mimo běžné pracovní prostředí -- věříme, že integrací těchto modelových příkladů můžeme motivovat studující k~průchodu kurzu.

\hypertarget{data-literacy-project}{%
\subsection{Data Literacy Project}\label{data-literacy-project}}

Zkouška~\parencite{course3}.

\part{Praktická část}

\hypertarget{koncept-kurzu}{%
\chapter{Koncept kurzu}\label{koncept-kurzu}}

Ve druhé části této bakalářské práce se budeme zabývat představením vlastního online kurzu zaměřeného na výuku základů datové analytiky. Ve dvou následujících kapitolách si popíšeme návrh a~implementaci našeho e-learningu z~hlediska dvou různých rovin, a~to konceptuální a~strukturální.

Našim cílem v~této části bude tedy jednak uvést motivaci a~cíle vztahující se k~tvorbě nového online kurzu, ale také přehledným způsobem popsat, z~jakých komponent se náš kurz skládá a~jaké vzdělávací prvky obsahuje. Na závěr také představíme jednotlivé obsahové moduly a~zarámujeme je do Bloomovy taxonomie vzdělávacích cílů.

\hypertarget{motivace}{%
\section{Motivace}\label{motivace}}

Online kurz vznikl pod záštitou firmy Digiskills\footnote{https://www.digiskills.cz/}, která se zabývá rozvojem digitálních kompetencí primárně v~korporátním prostředí. Portfolio služeb firmy Digiskills se dá vydělit do dvou hlavních částí. Jednak svým klientům nabízí takzvaný \emph{digiskills-assessment} -- jinými slovy audit digitálních dovedností, jehož cílem je zmapovat silné a~slabé stránky u~dané organizace a~nabídnout konkrétní aktivity na zlepšení digitálních kompetencí.

Druhým a~také hlavním produktem jsou pak online kurzy, které se převážně zaměřují na oblasti týkající se používání konkrétních nástrojů (jde například o, MS Teams, Office 365, Google Workspace nebo MS Power BI) a~na témata spojené s~digitálními kompetencemi (např. digitální produktivita, bezpečnost a~soukromí nebo efektivní práce z~domu). I~přes to, že některé tyto kurzy problematiku datové analytiky částečně reflektují, žádný z~nich ji nepojímá v~širším kontextu pro úplné začátečníky.

Proto vznikla motivace vytvořit v~českém prostředí takový online vzdělávací kurz, který se zabývá fundamentálními základy datové analytiky a~cílovým organizacím (případně jejich interním týmům anebo jednotlivým zaměstnancům) srozumitelně vysvětluje, k~čemu je datová analytika důležitá a~na základě jakých potřeb si je vhodné osvojit její základy.

Samotný kurz\footnote{V ekosystému kurzů firmy Digiskills se komplexnější online kurzy nazývají *Digitální akademie* – jde ale o~pouhou konvenci firmy, v~tomto textu na náš kurz referujeme obyčejným výrazem *online kurz*} vznikal za spolupráce s~firmou Digiskills, která nám při návrhu a~implementaci poskytla potřebnou infrastrukturu (primárně webové šablony a~technické vybavení) a~příležitostnou konzultaci, týkající se primárně charakteristiky cílové skupiny a~škálování technické náročnosti. Všechna ostatní rozhodnutí týkající se designu kurzu byla ponechána na nás.

\hypertarget{cuxedl}{%
\section{Cíl}\label{cuxedl}}

Při definování hlavního vzdělávacího cíle vycházíme z~revidované Bloomovy taxonomie kognitivních vzdělávacích cílů, která je přehledným způsobem uvedena v~\cite{vavra11} na základě kritiky\ldots{}

Hlavní cíl kurzu se dá vydělit do dvou hlavních dílčích cílů:

\begin{enumerate}
\def\labelenumi{\arabic{enumi}.}
\tightlist
\item
  student si je po skončení kurzu vědom, co je to datová analytika a~proč je důležité znát základní principy a~metody této disciplíny;
\item
  studující po skončení kurzu vědět, z~čeho skládá proces datové analytiky a~dokáže na základě těchto znalostí provést základní .
\end{enumerate}

\hypertarget{kompetence}{%
\section{Kompetence}\label{kompetence}}

\hypertarget{obsah-kurzu}{%
\chapter{Obsah kurzu}\label{obsah-kurzu}}

V~poslední kapitole si představíme vytvořený online kurz pro výuku datové analytiky. Tuto část dělíme do tří vzájemně provázaných částí, z~nichž každá reflektuje odlišnou oblast návrhu a~implementace online kurzu.

\hypertarget{prux16fchod}{%
\section{Průchod}\label{prux16fchod}}

V~první částí se zaměříme na průchod kurzem, popíšeme si tedy, jak je kurz navržen po strukturální stránce a~jakým způsobem student prochází jednotlivými částmi.

\textbackslash begin\{figure\}{[}ht{]}\\
\centering \textbackslash includegraphics{[}width={]}\{kurz-uvod\}\\

\caption{Úvodní obrazovka s~představením kurzu (v~prostředí Digiskills digitální akademie)}
    \label{kurz-uvod}

\textbackslash end\{figure\}

\hypertarget{komponenty}{%
\section{Komponenty}\label{komponenty}}

\hypertarget{moduly}{%
\section{Moduly}\label{moduly}}

\hypertarget{zuxe1vux11br}{%
\chapter*{Závěr}\label{zaver}\addcontentsline{toc}{chapter}{Závěr}}

Cílem této bakalářské práce bylo navrhnout a~implementovat online kurz zaměřený na výuku základů datové analytiky. Kurz se podařilo vytvořit dle zadaných požadavků a~v~aktuální chvíli probíhá uživatelské testování v~rámci interních klientů firmy Digiskills, s~níž jsme návrh kurzu konzultovali. Jelikož získávání zpětné vazby na vytvořený online kurz stále probíhá, nelze v~tuto chvíli diskutovat výsledky testování, je nicméně pravděpodobné, že se některé části kurzu mohou na základě získaných odpovědí v~průběhu nejbližší doby mírně změnit .

Na vytvořený online kurz lze dále navázat několika různými způsoby. Jelikož se jedná o~úvodní vstup do problematiky, může být e-learning rozšířen o~pokročilejší témata týkající se složitějších statistických přístupů, které lze aplikovat na větší soubor dat (tedy částečně propojit datovou analytiku s~disciplínou data science). Další cestou může být komplexnější představení konkurenčních analytických nástrojů, s~nimiž se student může setkat napříč odlišnými organizacemi. Třetí možnost je akcentovat témata související s~vizualizací dat a~směřovat studenta spíše do oblasti business analytics, v~rámci které je důležitější efektivnější komunikace informací, nežli provádění náročnějších analýz.

Cíle této práce byly splněny a~její výstupy můžou být rozšířeny v~rámci případného navazujícího aplikovaného výzkumu týkající se uživatelského testování.

\clearpage

\pagestyle{plain}

\addcontentsline{toc}{chapter}{Seznam použitých zdrojů}
\begin{spacing}{1.05}
\printbibliography[title={Seznam použitých zdrojů}]
\end{spacing}

\hypertarget{zuxe1vux11br}{%
\chapter*{Seznam příloh}\label{prilohy}\addcontentsline{toc}{chapter}{Seznam příloh}}

\begin{landscape}

\begin{table}[htbp]

\renewcommand\thetable{1}

\caption{\textit{Introduction to Data Analytics for Business}}\label{tab1}

\footnotesize

{

\justifying

\begin{tabularx}{\linewidth}{CCCCC}

\toprule

\textbf{Název} &

\textbf{Použitá platforma} &

\textbf{Datum vzniku} &

\textbf{Časová náročnost} &
 
\textbf{Specifikace obsahu kurzu}

\\

\tabularnewline
\midrule

Introduction to Data Analytics for Business

&

Coursera

&

1. 9. 2016

&

cca 12 hodin

&

Úvod do datové analytiky, základní koncepty decision making, datová analytika v~organizacích, základní techniky extrakce dat z~relačních databází

\\
\toprule

\textbf{Způsob práce s~hodnocením a~úkoly} &

\textbf{Forma ukončení} &

\textbf{Použité technologie} &

\textbf{Kompetenční rámec} 

\\

\tabularnewline
\midrule

Bez průběžných úkolů, testovací kvízy po každém z~modulů, finální praktický úkol založen na peer review

&

Finální známka je tvořena součtem dílčích známek z~jednotlivých kvízů (4 kvízy, váha 15 \%) a~zpracováním finálního praktického úkolu a~poskytnutý peer review (váha 40 \%)

&

Dotazovací jazyk SQL, relační databáze

&

Informační gramotnost – životní cyklus informací

Datová gramotnost – bezpečnost dat, manipulace s~daty

Datová analytika – analytika v~reálném světě, komunikace analytických výsledků 

\\

\tabularnewline
\bottomrule
\end{tabularx}

}

\end{table}

\end{landscape}


\end{document}