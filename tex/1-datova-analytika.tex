\part{Teoretická část}

\hypertarget{datovuxe1-analytika}{%
\chapter{Datová analytika}\label{datovuxe1-analytika}}

Ústředním tématem této práce je fenomén datové analytiky, jejímž
význačným rysem je vágnost přesné definice, protože se na ni v~odlišných
prostředích a~kontextech nahlíží různým pohledem dle daných potřeb.
Abychom byli schopni provést systematickou přehledovou studii online
kurzů zabývajících se touto problematikou a~zároveň při návrhu vlastního
e-learningového řešení vycházet z~určitého kompetenčního rámce, se
v~této kapitole pokusíme tuto disciplínu zarámovat do konceptu datové
gramotnosti. Ve druhé části této sekce se pak zaměříme na konkrétní
kompetence, jež jsou součástí datové analytiky, a~z~nichž budeme
v~dalších částech vycházet.

\hypertarget{datovuxe1-gramotnost}{%
\section{Datová gramotnost}\label{datovuxe1-gramotnost}}

Při práci s~konceptem datové gramotnosti se můžeme setkat s~různými
pojetími, které se buď vzájemně překrývají, zastupují či doplňují. Dle
některých autorů lze datovou gramotnost zahrnout pod gramotnost
informační, protože právě ta může ve své obecné definici zahrnovat
i~práci s~daty s~jakožto speciálním druhem informace.
\parencite[126]{calzada13} Například dle standardu instituce
ACRL\footnote{The Association of College and Research Libraries – [http://www.ala.org/acrl/](http://www.ala.org/acrl/)}
\emph{Information Literacy Competency Standards for Higher Education} je
informační gramotnost definována jako souhrn schopností a~dovedností
jedince, které slouží k~identifikaci své informační potřeby a~jejímu
následnému uspokojení protřednictvím lokalizace, evaluace (zhodnocení)
a~efektivního využití dané informace. \parencite[2]{acrl06} Podobným
způsobem definuje informační gramotnost
i~\href{https://tdkiv.nkp.cz/}{TDKIV}: ‚‚Schopnost jedince identifikovat
informační potřebu, vyhledat informace, zhodnotit je, zpracovat
a~efektivně využít.`\,` \parencite{tdkiv03}

Jiní autoři pracují s~významem datové gramotnosti ve spojitosti se
statistickou gramotností, resp. uvažují informační gramotnost jako
prerekvizitu k~datové gramotností, jelikož práce s~daty vyžaduje
kritický pohled na informace -- jejích správně čtení, hodnocení
a~interpretaci. Bez těchto složek si lze pak těžko představit datovou
gramotnost jakožto proces skládající se z~těchto části:
\parencite[8]{schield05} - přístup k~datům; - posouzení kvality dat; -
manipulace s~daty; - sumarizace dat; - prezentace dat.
