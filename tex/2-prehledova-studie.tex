\hypertarget{pux159ehledovuxe1-studie-online-kurzux16f}{%
\chapter{Přehledová studie online kurzů}\label{pux159ehledovuxe1-studie-online-kurzux16f}}

\hypertarget{charakteristika-pux159ehledovuxe9-studie}{%
\section{Charakteristika přehledové studie}\label{charakteristika-pux159ehledovuxe9-studie}}

Ve druhé kapitole teoretické části se budeme zabývat již existujícími online kurzy, které se určitým způsobem soustředí na problematiku datové analytiky či alespoň zpracovávají některou z~jejích vybraných kompetencí. Našim cílem je v~této sekci systematicky kategorizovat významná e-learningová řešení a~na základě této klasifikace vytvořit určitou formu přehledové studie\footnote{Je si však zapotřebí uvědomit, že svojí povahou je tato bakalářská práce nejblíže aplikační kvalifikační práci, jejímž cílem bylo vytvořit funkční online kurz, z~tohoto důvodu nelze považovat tuto přehledovou studii za plnohodnotnou výzkumnou studii, tak jak je uváděna v~odborné literatuře~\parencite{mares2013}.}.

Výsledky získané z~této přehledové studie posléze využijeme k~návrhu jednotlivých částí našeho online kurzu, který je zaměřen na výuku základů datové analytiky primárně v~podnikovém prostředí (vlastní řešení konkrétněji rozvádíme v~praktické části této práce).

Každý zpracovaný online kurz (vybírali jsme pouze takové kurzy, s~jejichž obsahem se dá alespoň částečně bezplatně pracovat, a~mají tedy nějakou podobu otevřeného přístupu) sestává ze tří oddělených ale vzájemně se doplňujících částí.

\hypertarget{anotace-kurzu-s-odux16fvodnux11bnuxedm-vuxfdbux11bru}{%
\subsection{Anotace kurzu s~odůvodněním výběru}\label{anotace-kurzu-s-odux16fvodnux11bnuxedm-vuxfdbux11bru}}

V~rámci anotace popisujeme základní cíle kurzu, ve stručnosti charakterizujeme jeho obsahovou stránku a~nastiňujeme strukturu, z~níž se daný kurz skládá. Dále se snažíme zdůvodnit, proč jsme daný kurz vybrali, a~případně se jej pokoušíme začlenit do některého z~konceptů informační či datové gramotností, popřípadě do dalších kompetenčních rámců.

\hypertarget{tabulka-s-parametry}{%
\subsection{Tabulka s~parametry}\label{tabulka-s-parametry}}

Další částí analýzy u~každého z~online kurzů je tabulka s~předem definovanými parametry, která systematicky uvádí základní informace a~přehledným způsobem znázorňuje charakteristiku online kurzu. Konkrétní tabulku s~parametry k~danému kurzu vždy přikládáme do seznamu příloh a~prostřednictvím odkazu propojujeme s~anotací. Níže uvádíme seznam vybraných parametrů:

\begin{itemize}
\tightlist
\item
  název;
\item
  použitá platforma;
\item
  datum vzniku;
\item
  specifikace obsahu kurzu (jaká témata kurz reflektuje);
\item
  časová náročnost;
\item
  způsob práce s~hodnocením a~úkoly;
\item
  forma ukončení;
\item
  začlenění do kompetenčního rámce;
\item
  použité technologie.
\end{itemize}

\hypertarget{popis-vyuux17eituxfdch-prvkux16f}{%
\subsection{Popis využitých prvků}\label{popis-vyuux17eituxfdch-prvkux16f}}

Poslední složkou přehledu je popis prvků, které jsme se rozhodli integrovat do našeho praktického řešení. V~této části se nevážeme pouze na jednotlivá témata či použité nástroje, ale všímáme si také zvolených edukačních prostředků. Může jít tedy například o~způsoby průběžného ověřování znalosti a~dovedností, zvolené aktivizační metody za účelem větší interaktivity nebo možnosti získávání zpětné vazby na jednotlivé obsahové moduly. Tyto prvky nám dále poslouží k~návrhu vlastních vzdělávacích modulů našeho online kurzu, případně k~redesignu již existujících komponent.

\hypertarget{online-kurzy}{%
\section{Online kurzy}\label{online-kurzy}}

\hypertarget{introduction-to-data-analytics-for-business}{%
\subsection{Introduction to Data Analytics for Business}\label{introduction-to-data-analytics-for-business}}

\hypertarget{anotace}{%
\subsubsection{Anotace}\label{anotace}}

Kurz \emph{Introduction to Data Analytics for Business}\footnote{Systematickou analýzu kurzu uvádíme v~tabulce s~parametry viz Tabulka \ref{tab1}.} na platformě Coursera je součástí širší specializace pod názvem \emph{Advanced Business Analytics Specialization} a~je zprostředkován pod záštitou University of Colorado Boulder. Tento kurz je úvodní částí zmíněné specializace a~jeho cílem je popsat základní principy datové analytiky v~podnikovém prostředí. Obsah kurzu se velkou mírou soustředí na integraci datové analytiky do procesů určité organizace a~na konkrétních příkladech ukazuje, proč je důležité tuto oblast reflektovat na individuální úrovni ve smyslu zavádění analytických technik do vlastních pracovních mikroprocesů, a~tak zvyšovat celo-organizační efektivitu. Kurz každopádně nezůstává pouze u~teoretických základů, ale v~pozdějších částech pracuje s~využitím konkrétních nástrojů a~představuje základní analytické techniky na úrovni relačních databází a~dotazovacího jazyku SQL~\parencite{course1}.

Struktura kurzu je tvořena čtyřmi na sebe navazujícími sekcemi:

\begin{itemize}
\tightlist
\item
  Data and Analysis in the Real World;
\item
  Analytical Tools;
\item
  Data Extraction Using SQL;
\item
  Real World Analytical Organizations.
\end{itemize}

Tento kurz jsme do našeho výběru zahrnuli jednak proto, že pracuje s~konceptem jednoho komplexního úkolů napříč různými tématy, na němž si studenti mají možnost vyzkoušet všechny potřebné metody, a~jednak z~toho důvodu, že se nezaměřuje pouze na představení konkrétních analytických nástrojů, ale snaží se reflektovat motivaci, která je příčinou pro využívání datové analytiky v~podnikovém prostředí.

\hypertarget{vyuux17eituxe9-prvky}{%
\subsubsection{Využité prvky}\label{vyuux17eituxe9-prvky}}

Za hodnotnou část tohoto kurzu vnímáme primárně teoretičtější téma využití datové analytiky v~podnikovém prostředí -- konkrétně jde tedy o~první a~čtvrtý modul. Autoři kurzu totiž přehledným způsobem vysvětlují datovou analytiku jakožto nutnou součást firemních procesů, které začínají u~samotných jednotlivců a~časem přesahují do vyšších pater dané organizace. Tuto část pokládáme ze velmi důležitou a~pokusíme se ji tak začlenit do úvodních částí našeho vlastního kurzu.

Druhou složkou, jež bychom rádi nějakým způsobem integrovali, je koncept jednoho komplexního úkolů, na němž studenti po celou dobu studia pracují. V~tomto případě šlo sice o~úkol, který se nacházel až v~posledním modulu, nicméně sestával ze všech předcházejících dovedností a~pracoval s~reálným podnikovým modelem. Rádi bychom tedy i~vlastní studenty pobídli k~práci na jednom větším projektu, který se skládá ze vzájemně na sebe navazujících části a~který je co nejvíce přibližuje k~reálným procesům určité organizace.

\hypertarget{intro-to-data-analysis}{%
\subsection{Intro to Data Analysis}\label{intro-to-data-analysis}}

\hypertarget{anotace-1}{%
\subsubsection{Anotace}\label{anotace-1}}

Druhý analyzovaný kurz je součástí většího balíku kurzů na online platformě Udacity, které společně tvoří takzvaný \emph{Data Analyst Nanodegree}. Jedná se o~ucelený celek vzdělávacích kurzů, které si kladou za cíl vzdělat studenta na požadovanou úroveň v~dané oblasti pro účely větší konkurenceschopnosti na trhu práce. Výhodou toho programu je nabytí praktické zkušenosti s~konkrétními analytickými nástroji a~technikami, nicméně to vše za cenu poměrně vysokého finančního poplatku.

My se každopádně zaměřujeme na kurz \emph{Intro to Data Analysis}\footnote{Systematickou analýzu kurzu uvádíme v~tabulce s~parametry viz Tabulka \ref{tab2}.}, který je bezplatným úvodním seznámením s~tématem datové analytiky v~rámci celé této specializace. I~přesto je však tento kurz zaměřen převážně prakticky, protože se sám dělí ještě do tří vzájemně navazujících lekcí (spolu se závěrečným úkolem), z~nichž se právě dvě zabývají poměrně pokročilými nástroji (jde o~programovací jazyk Python a~jeho dvě knihovny používané pro statistické účely Numpy a~Pandas).

Pro naše účely je však důležitá primárně první lekce, jež se zabývá kontextualizací datové analytiky jakožto samostatné disciplíny a~která přehledným způsobem demonstruje všechny potřebné složky, které jsou součástí procesu datové analytiky. V~pojetí tohoto prakticky zaměřeného kurzu jde tedy o: \emph{question} (jaký problém chceme analýzou řešit), \emph{wrangle} (přístup k~datům a~jejich čištění), \emph{explore} (vlastní explorativní analýza na základě očištěných dat), \emph{draw conclusions} (získání nové informace) a~\emph{communicate} (komunikace výsledné informace prostřednictvím vizualizace)~\parencite{course2}.

\hypertarget{vyuux17eituxe9-prvky-1}{%
\subsubsection{Využité prvky}\label{vyuux17eituxe9-prvky-1}}

I~přes to, že svým praktickým pojetím tento kurz značně přesahuje potřeby, které jsou kladeny na naše vlastní řešení, lze využít použitého konceptu procesu datové analytiky. Autoři kurzu tímto způsobem efektivně vydělili a~zpracovali jednotlivé kroky, které vedou k~úspěšnému provedení datové analytiky -- my tak můžeme pro naši praktickou část podobně použít obsah dokumentů \emph{ECDL / ICDL Data Analytics SYLABUS 1.0 (AM8)} (viz kapitola \ref{kompetenux10dnuxed-ruxe1mec}) a~přetvořit jej do podobně strukturované podoby, jako byla použita v~tomto kurzu.

Druhý významný přínos je pro nás ve srozumitelném představení základních příkladů, kdy můžeme datovou analytiku využít na dennodenní bází mimo běžné pracovní prostředí -- věříme, že integrací těchto modelových příkladů můžeme motivovat studující k~průchodu kurzu.

\hypertarget{data-literacy-project-overview-of-data-literacy-data-fundamentals}{%
\subsection{Data Literacy Project -- Overview of Data Literacy, Data Fundamentals}\label{data-literacy-project-overview-of-data-literacy-data-fundamentals}}

\hypertarget{anotace-2}{%
\subsubsection{Anotace}\label{anotace-2}}

Třetí analyzovaný e-learning není ani tak přímo jeden jediný online kurz, nýbrž jde o~kolekci většího množství online materiálů. Jedná se o~\emph{Data Literacy Project}\footnote{Systematickou analýzu kurzu uvádíme v~tabulce s~parametry viz Tabulka \ref{tab3}.}, vzdělávací platformu pod záštitou organizace Qlik, která si klade za cíl edukovat veřejnost v~oblasti datové gramotnosti. Tento projekt přichází se svým vlastním pojetím datové gramotnosti a~člení ji do několika oddělených podkurzů -- pro nás jsou relevantní primárně tyto dvě oblasti (v~terminologii projektu se s~nimi pracuje právě na úrovní jednotlivých subkurzů):

\begin{itemize}
\tightlist
\item
  Overview of Data Literacy;

  \begin{itemize}
  \tightlist
  \item
    Why Analytics?;
  \item
    A~Culture of Data Literacy;
  \item
    Data Literacy Adoption;
  \item
    Data Storytelling;
  \end{itemize}
\item
  Data Fundamentals;

  \begin{itemize}
  \tightlist
  \item
    Understanding Data;
  \item
    Understanding Aggregations;
  \item
    Understanding Distributions~\parencite{course3}.
  \end{itemize}
\end{itemize}

Tento kurz jsme vybrali hlavně ze dvou důvodů. Jednak proto, že zpracovává určitá dílčí témata datové analytiky v~kontextu datové gramotnosti (např. charakteristiku dat jakožto objektů, kterými se datová analytika zabývá) a~jednak z~toho důvodu, že originálním způsobem představuje možnosti datové gramotnosti a~analytiky v~organizacích. Resp. jsou v~online materiálech zevrubně popsány jednotlivé kroky (ty na sebe cyklicky navazují -- plánování a~vize, komunikace, zhodnocení, vzdělávání s~evaluace), kterými se je zapotřebí řídit, chceme-li svoji organizaci nebo podnik přetvořit v~datově orientované prostředí, kde se rozhodnutí vytváří na datově podloženém základě.

\hypertarget{vyuux17eituxe9-prvky-2}{%
\subsubsection{Využité prvky}\label{vyuux17eituxe9-prvky-2}}

Pro návrh vlastního e-learningového řešení můžeme z~tohoto projektu využít části týkající se datově orientované organizace, protože lze pod tímto konceptem zastřešit důležitost vzdělávání v~oblasti datové gramotnosti a~analytiky v~podnikovém prostředí. Taktéž můžeme převzít příklady, na nichž autoři charakterizovali jednotlivé typy dat, a~tím demonstrovat různorodost analytických technik, které jsou závislé na určitém typu dat.

\hypertarget{essentials-of-data-literacy}{%
\subsection{Essentials of Data Literacy}\label{essentials-of-data-literacy}}

\hypertarget{anotace-3}{%
\subsubsection{Anotace}\label{anotace-3}}

Kurz \emph{Essentials of Data Literacy}\footnote{Systematickou analýzu kurzu uvádíme v~tabulce s~parametry viz Tabulka \ref{tab4}.} spadá do rodiny online kurzů platformy edX a~je součástí profesionální certifikátu \emph{Fundamentals of Data Visualization with Power BI}. Již z~názvu certifikátu je patrné, že se jedná o~primárně praktický kurz, jehož cílem je studenty naučit efektivně analyzovat a~vizualizovat data a~na základě nich komunikovat určitou informaci. Obsahově je kurz zaměřen spíše na akademické prostředí, protože se snaží pobídnout studenty a~další akademické pracovníky z~různých oborů k~datově orientovanému stylu práce nehledě na studovaný obor. Autoři kurzu představují koncept životního cyklu informace, který se skládá z~šesti cyklicky opakujících se fází -- definování otázky, sběr a~organizace dat, čištění dat, explorace a~vizualizace dat, analýza a~interpretace dat, komunikace prostřednictvím nasbíraných dat. Tohle teoretické pojetí umožňuje autorům dále pracovat s~jednotlivými fázemi v~praktičtější rovině, protože se po zbytek kurzu věnují problematice zpracování dat v~programovacím jazyce R~\parencite{course4}.

Hodnotu tohoto e-learningového řešení spatřujeme primárně ve funkční provázanosti teoretického rámce s~praktickými sekcemi -- autoři kurzu tak mají naprostou většinu úkolů zdůvodněnou konkrétními potřebami a~dokážou díky výše zmíněnému rámci pěkně kontextualizovat zdánlivě nesouvisející oblasti.

\hypertarget{vyuux17eituxe9-prvky-3}{%
\subsubsection{Využité prvky}\label{vyuux17eituxe9-prvky-3}}

Jak bylo výše poznamenáno, tento kurz je příkladem správného koncepčního uvažování při návrhu vzdělávacích materiálů, kdy autoři nejprve dbaly na úvodní motivaci a~teorii, kterou postupně funkčními přístupy (přehlednými autentickými videy a~srozumitelnými textovými materiály) transformovali do praktických úkolů. Rádi bychom tak využili tohoto přístupu při návrhu vlastního řešení a~soustředili se na neustále propojování nově nabytých dovednosti/znalosti s~těmi, které jim buď předcházely nebo se mají teprve objevit.

\hypertarget{analyzing-and-visualizing-data-with-power-bi}{%
\subsection{Analyzing and Visualizing Data with Power BI}\label{analyzing-and-visualizing-data-with-power-bi}}

\hypertarget{anotace-4}{%
\subsubsection{Anotace}\label{anotace-4}}

Další zpracovaný kurz \emph{Analyzing and Visualizing Data with Power BI}\footnote{Systematickou analýzu kurzu uvádíme v~tabulce s~parametry viz Tabulka \ref{tab5}.} svojí formou částečně souvisí s~předcházejícím kurzem, jelikož je taktéž součástí zmíněného profesního certifikátu. Obsahově se ale týká odlišené oblasti, protože komplexním způsobem zpracovává téma datové analytiky (je zde patrný překryv s~\emph{business analytics}) v~kontextu analytického nástroje Power BI. Kurz je tedy navržen primárně s~ohledem na praktickou stránku věci a~studenta tak provází celkovou tvorbou interaktivních dashboardů. Tento přístup je ale zároveň výhodou i~nevýhodu současně. Na jednu stranu kurz sice nabízí ucelený přehled funkcionalit jednoho konkrétního nástroje, ale na stranu druhou opomíjí skutečnost, že vybraný analytický nástroj Power BI je závislý pouze na systémech s~operačním systémem Windows a~může být tak problematické využít tyto nabízené dovednosti i~v~jiných prostředích~\parencite{course5}. Obsahově se kurz dělí do čtyř na sebe navazujících částí:

\begin{itemize}
\tightlist
\item
  Getting Started -- stručný úvod a~základní seznámení s~importem a~čištěním dat;
\item
  Sports Analytics and Global Economic Indicators -- pokročilé techniky čištění dat a~základní vizualizační techniky;
\item
  Personal Finance -- pokročilé analytické funkce, základní statistické metody a~pokročilé vizualizační techniky;
\item
  Power BI Service and Mobile App -- kontextualizace s~jinými nástroji, komplexní zadání závěrečné práce.
\end{itemize}

Tento kurz jsme vybrali hlavně kvůli jeho podobě a~použitým vzdělávacích prvkům, za hodnotnou složku tedy primárně považujeme formu, jakou představuje určitou práci s~praktickým nástrojem. Autorský tým dokázal prostřednictvím edukačních videí přehledným způsobem krok za krokem představovat základní principy softwaru Power BI a~posouvat tak studentovy dovednosti a~znalosti stále ke složitějším manipulacím s~daty.

\hypertarget{vyuux17eituxe9-prvky-4}{%
\subsubsection{Využité prvky}\label{vyuux17eituxe9-prvky-4}}

I~přes to, že se náš vlastní online kurz nemá vázat na jeden konkrétní analytický nástroj, chceme v~hlavní části demonstrovat základní principy datové analytiky na základě praktických ukázek. Nevyhneme se proto práci s~nějakým často používaným softwarem. Z~tohoto důvodu můžeme využít přístup tohoto e-learningového řešení k~předávání praktických dovedností prostřednictvím video materiálů. Jak bylo výše zmíněno, tento kurz kvalitní formou představuje základní i~pokročilé techniky v~Power BI, my se tak v~praktické části můžeme inspirovat některými prvky jako jsou například interaktivní kvízy, které doplňují obsah videí, nebo kombinací snímání obrazovky a~krátkých přednášek lektora apod.

\hypertarget{introduction-to-data-analytics}{%
\subsection{Introduction to Data Analytics}\label{introduction-to-data-analytics}}

Další analyzovaný online kurz \emph{Introduction to Data Analytics}\footnote{Systematickou analýzu kurzu uvádíme v~tabulce s~parametry viz Tabulka \ref{tab6}.} spadá pod platformu Coursera a~je součástí čtyř na sebe navazujících e-learningů, které jsou zarámované profesním certifikátem \emph{IBM Data Analyst Professional Certificate} (certifikát se snaží reflektovat požadavky na pracovní trh, u~něhož se dle predikcí předpovídá do roku 2028 až 20\% nárust pozic spojených s~datovou analytikou\footnote{https://www.coursera.org/professional-certificates/ibm-data-analyst}). Tento kurz je úvodní částí a~zaměřuje se tak na základní teoretické koncepty týkající se datové analytiky. Velký důraz klade na popis role datového analytika -- snaží se popsat co je náplní jeho práce (a~zároveň tuto roli představuje v~kontextu dalších pracovních rolí jako jsou data scientist a~data engineer) a~představuje s~jakými nástroji a~technikami by měl být schopen pracovat~\parencite{course6}.

Ve druhé části je pak přehlednou formou vysvětlen proces datové analytiky skládající se z:

\begin{itemize}
\tightlist
\item
  identifikace zdrojů dat a~jejich sběr;
\item
  čištění a~vlastní analýza dat;
\item
  tvorba vizualizací a~komunikace dat.
\end{itemize}

I~přes to, že se kurz v~některých podkapitolách zaměřuje na praktická témata (primárně u~jednotlivých částí procesu, ale i~třeba při popisu využívaných nástrojů), nedemonstruje žádné techniky zpracování dat na konkrétních příkladech. Představuje tedy obecně platné principy, které je zapotřebí brát do úvahy při vlastní praxi, a~právě díky této nezávislosti na konkrétnímu nástroji, jsme tento kurz začlenili do našeho výběru.

\hypertarget{vyuux17eituxe9-prvky-5}{%
\subsubsection{Využité prvky}\label{vyuux17eituxe9-prvky-5}}

Autoři kurzu si dali záležet na častém používání doprovodných kvízů téměr po každé samostatné obsahové části. Jelikož považujeme za důležité pracovat s~neustálou aktivizací studujícího, můžeme využít některých již vytvořených kvízu a~adaptovat je na naše konkrétní potřeby. Druhou významnou složkou, kterou bychom určitým způsobem integrovali do našeho řešení, je právě srozumitelné vysvětlení role datového analytika v~rámci pracovního trhu a~porovnání jeho profesní náplně s~příbuznými oblastmi.

\hypertarget{excel-basics-for-data-analysis}{%
\subsection{Excel Basics for Data Analysis}\label{excel-basics-for-data-analysis}}

\hypertarget{anotace-5}{%
\subsubsection{Anotace}\label{anotace-5}}

\hypertarget{vyuux17eituxe9-prvky-6}{%
\subsubsection{Využité prvky}\label{vyuux17eituxe9-prvky-6}}

\hypertarget{data-analysis-with-excel-for-complete-beginners}{%
\subsection{Data Analysis with Excel for Complete Beginners}\label{data-analysis-with-excel-for-complete-beginners}}

\hypertarget{anotace-6}{%
\subsubsection{Anotace}\label{anotace-6}}

\hypertarget{vyuux17eituxe9-prvky-7}{%
\subsubsection{Využité prvky}\label{vyuux17eituxe9-prvky-7}}
