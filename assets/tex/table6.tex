\begin{landscape}

\begin{table}[htbp]

\renewcommand\thetable{7}

\caption{\textit{Introduction to Data Analytics}}\label{tab6}

\footnotesize

{

\justifying

\begin{tabularx}{\linewidth}{CCCCC}

\toprule

\textbf{Název} &

\textbf{Použitá platforma} &

\textbf{Datum vzniku} &

\textbf{Časová náročnost} &
 
\textbf{Specifikace obsahu kurzu}

\\

\tabularnewline
\midrule

Introduction to Data Analytics

&

Coursera

&

1. 9. 2020

&

cca 11 hodin v~rámci 4 týdnů

&

Role datového analytika, povaha a~typ dat, základní zpracování dat – import, čištění, analýza a~vizualizace

\\
\toprule

\textbf{Způsob práce s~hodnocením a~úkoly} &

\textbf{Forma ukončení} &

\textbf{Použité technologie} &

\textbf{Kompetenční rámec} 

\\

\tabularnewline
\midrule

Průběžné kvízy (cca 2–3 na týden) a~jeden závěrečný úkol    

&

Pro úspěšné ukončení kurzu je zapotřebí splnit všechny průběžné kvízy a~vypracovat finální teoretický úkol, který bude posléze zhodnocen ostatními studenty

&

–

&

Datová gramotnost – přístup k~datům a~jejich sběr, prezentace výsledků získaných ze zpracovaných dat

Datová analytika – ukotvení role datového analytika v~kontextu pracovnách rolí data scientist a~data engineer, teoretické uvedení disciplíny  

\\

\tabularnewline
\bottomrule
\end{tabularx}

}

\end{table}

\end{landscape}