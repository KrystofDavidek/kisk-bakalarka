\part{Teoretická část}

\hypertarget{datovuxe1-analytika}{%
\chapter{Datová analytika}\label{datovuxe1-analytika}}

Ústředním tématem této práce je fenomén datové analytiky, jejímž význačným rysem je vágnost přesné definice, protože se na ni v~odlišných prostředích a~kontextech nahlíží různým pohledem dle daných potřeb. Abychom byli schopni provést systematickou přehledovou studii online kurzů zabývajících se touto problematikou a~zároveň mohli při návrhu vlastního e-learningového řešení vycházet z~určitého kompetenčního rámce, se v~této kapitole pokusíme tuto disciplínu zarámovat do konceptů datové a~digitální gramotnosti. Ve druhé části této sekce se pak zaměříme na konkrétní kompetence, jež jsou součástí datové analytiky, a~z~nichž budeme v~dalších částech vycházet.

\hypertarget{datovuxe1-gramotnost}{%
\section{Datová gramotnost}\label{datovuxe1-gramotnost}}

Při práci s~konceptem datové gramotnosti se můžeme setkat s~různými pojetími, které se buď vzájemně překrývají, zastupují či doplňují. Dle některých autorů lze datovou gramotnost zahrnout pod gramotnost informační, protože právě ta může ve své obecné definici zahrnovat i~práci s~daty s~jakožto speciálním druhem informace.~\parencite[126]{calzada13} To znamená, že lze informační gramotnost uvažovat jako prerekvizitu k~datové gramotností, jelikož práce s~daty vyžaduje kritický pohled na informace -- jejich správně čtení, hodnocení a~interpretaci.

Dle uznávaného standardu instituce \emph{ACRL}\footnote{The Association of College and Research Libraries – http://www.ala.org/acrl/} je informační gramotnost definována jako souhrn schopností a~dovedností jedince, které slouží k~identifikaci své informační potřeby a~jejímu následnému uspokojení protřednictvím lokalizace, evaluace (vyhodnocení) a~efektivního využití dané informace.~\parencite[2]{acrl14} Podobným způsobem definuje informační gramotnost i~TDKIV\footnote{https://tdkiv.nkp.cz/}: \uv{Schopnost jedince identifikovat informační potřebu, vyhledat informace, zhodnotit je, zpracovat a~efektivně využít.}~\parencite{tdkiv03}

Bez těchto složek si lze pak těžko představit datovou gramotnost tak, jak je představena ve svém nejobecnějším pojetí:~\parencite[8]{schield05}

\begin{itemize}
\tightlist
\item
  přístup k~datům a~jejich sběr;
\item
  posouzení kvality dat;
\item
  manipulace s~daty a~jejich zpracování;
\item
  sumarizace a~zhodnocení dat;
\item
  prezentace dat.
\end{itemize}

Další autoři pracují s~významem datové gramotnosti ve spojitosti se statistickou gramotností. Pod touto gramotností si lze představit dovednosti spojené se čtením, interpretací a~tvorbou statistických dat jako jsou grafy, tabulky a~číselná data (obecněji řečeno jde o~základní znalost deskriptivní statistiky).~\parencite[8]{schield05}. Nicméně my se budeme v~této práci držet názorového proudu, který tyto specifické dovednosti často techničtější povahy chápe jako podsoučást datové gramotnosti.~\parencite[125]{calzada13}

Prováznost informační a~datové gramotnosti můžeme dále najít ve standardu informační gramotnosti \emph{The seven pillars of information literacy}, který inherentně předpokládá, že právě data jsou s~informacemi vzájemně spjaty, a~proto lze s~nimi pracovat na principiálně podobné, ne-li stejné úrovni.~\parencite[126]{calzada13} Tento standard nabízí sedm pilířů (oblastí), které je nutné v~rámci práce s~informacemi/daty provádět.~\parencite{sconul11} Předchozí výčet dovedností a~schopností bychom tak mohli modifikovat následovným způsobem:

\begin{itemize}
\tightlist
\item
  identifikace datové potřeby;
\item
  zhodnocení aktuálních znalostí a~případné zjištění nedostatků;
\item
  vytvoření strategie pro vyhledání potřebných dat;
\item
  přístup k~datům a~jejich sběr;
\item
  evaluace kvality získaných dat;
\item
  etická manipulace s~daty spolu s~jejich zpracováním;
\item
  prezentace výsledků získaných ze zpracovaných dat.
\end{itemize}

V~literatuře se lze dále setkat s~konceptem \emph{Data information literacy} (DIL), který sice úzce souvisí s~informační (resp. datovou) gramotností, nicméně se s~ním pracuje primárně v~rámci zpracování vědeckých a~vzdělávacích dat.~\parencite{jeffryes13} Jde tedy především o~kompetence spojené se správou výzkumných dat, nicméně i~v~této oblasti lze najít přesahy do některých podoblastí popsaných výše.

Například s~problematikou vizualizace dat lze pracovat jak v~DIL, tak i~na úrovní ostatních kompetenčních rámců, jimž může být třeba již zmíněný standard \emph{Information literacy competency standards for higher education}. Resp. jednotlivé moduly tohoto standardu v~sobě implicitně obsahují témata související s~vizualizací dat -- najít je můžeme konkrétně ve druhém (posouzení), třetím (evaluace) a~primárně čtvrtém (využití) modulu. Důležitým aspektem je pak pasivní přítomnost tohoto tématu napříč kompetencemi, protože ku příkladu ve druhém modulu se předpokládá, že právě ten jedinec, který se dokáže vyznat a~orientovat ve vizuálně laděných informací, může splňovat širší kompetenci týkající se správného zhodnocení informací.~\parencite{womack14} Věříme proto, že jsme schopni naprostou většinu témat, ze kterých sestává oblast datové analytiky (např. již zmíněná vizualizace dat) hledat právě v~těchto standardech, a~je z~nich tedy možné v~obecnějším pojetí vycházet.

V~odborném prostředí si také dále můžeme všimnout tendenci k~edukaci datové gramotnosti prostřednictvím výuky, která u~studentů podporuje rozhodování založených na datech (\emph{data-driven decisions}).~\parencite{mandinach13} Pěkným příkladem může být propojení tohoto přístupu s~konkrétními problémy pocházející z~reálného světa, kde mají studenti všechny principy související s~\emph{data-driven decisions} kombinovat s~tzv. \emph{problem solving} technikami. A~to vše ve spojitosti se zpracováním tabulkových dat v~určitém tabulkovém procesoru.~\parencite{slayter17}

\hypertarget{digituxe1lnuxed-gramotnost}{%
\section{Digitální gramotnost}\label{digituxe1lnuxed-gramotnost}}

Druhým neméně důležitým druhem gramotnosti, který s~našim tématem úzce souvisí, je gramotnost digitální. Tento koncept je dobře popsán ve dvou stěžejních standardech/certifikátech -- jsou jimi evropský rámec \emph{DIGCOMP 2.0}\footnote{https://ec.europa.eu/jrc/en/digcomp/digital-competence-framework} a~mezinárodně standardizované sylaby konceptu \emph{ECDL / ICDL}\footnote{https://icdleurope.org/} (dříve \emph{European Computer Driving Licence}, dnes již přejmenován na \emph{European certification of Digital Literacy}).

Do jisté míry zjednodušení se dá říci, že standard \emph{DIGCOMP 2.0} nabízí potřebný výčet digitální kompetencí spolu s~konkrétními možnostmi realizace\footnote{Jedna se o~pět na sebe navazujících modulů – informační a~datová gramotnost, komunikace a~spolupráce, tvorba digitálního obsahu, bezpečnost, schopnost řešení problémů~\parencite{digicomp17}} a~tyto kompetence se snaží uvést v~univerzální vzdělávací rámec. Na ten pak částečně navazuje koncept \emph{ECDL / ICDL}, jenž v~sobě mimo jiné obsahuje samostatný program \emph{ECDL DIGCOMP}, prostřednictvím kterého lze tyto kompetence ověřit platným certifikátem.~\parencite{chabera21}

V~kontextu obou standardů si můžeme všimnout, že se problematika informační a~datové gramotnosti (a~jejich dalších podkomponent) objevuje na větším množství míst. To, co je ale pro naši charakterizaci datové analytiky významnější, jsou certifikáty \emph{ECDL Advanced}\footnote{https://www.ecdl.cz/cert\_ecdl\_advanced.php}, které obsahují vzdělávací sylaby znalostí a~dovedností potřebných pro získání daného certifikátu. Obsah těchto sylabů je totiž určen pro digitálně kvalifikovanou veřejnost a~pokrývá profesionální uživatelské znalosti a~dovednosti v~oblasti kancelářských aplikací.~\parencite{ecdl17} Obsah tohoto modulu se v~další podkapitole pokusíme zasadit do kontextu požadavků na oblast datové analytiky.

\hypertarget{kompetence-datovuxe9-analytiky}{%
\section{Kompetence datové analytiky}\label{kompetence-datovuxe9-analytiky}}

Jedním z~přístupů při hledání konkrétních kompetencí datové analytiky může být porovnání této disciplíny s~oblastmi příbuznými, jako jsou primárně data science a~business analytics\footnote{U těchto termínů budeme v~textu používat jejich anglické varianty, protože se tak používají i~rámci českého prostředí. Na druhou stranu u~datové analytiky namísto data analytics volíme českou variantu, protože se náš kurz vztahuje k~českému prostředí, kde chceme tuto oblast popularizovat.}. V~odborných pracích se k~hledání podobností a~odlišností napříč těmito disciplínami často využívá metoda obsahové analýzy, prostřednictvím které lze tyto vztahy zachytit a~popsat. My se zde v~následujících řádcích pokusíme ve stručnosti sumarizovat některé závěry, které z~těchto studií vychází a~které považujeme za relevantní pro naše téma.

První studie se zabývala porovnáním datové analytiky s~doménou data science v~kontextu nabídky univerzitních kurzů. Práce tedy zkoumá obsahy vybraných studijních programů a~snažila se je charakterizovat prostřednictvím popisu jednotlivých dovedností/schopností. Z~textu vyplývá, že hlavní podobnost těchto dvou oborů je v~zakomponování témat vizualizace dat a~práce s~numerickými a~matematickými operacemi. Zajímavostí je, že stejně tak v~obsahu kurzů chybí u~obou oblastí akcentování tématu datové etiky či správy velkých dat.~\parencite[109]{aasheim2015}

Co se týká rozdílných kompetencí, tak hlavní odlišností je, že v~rámci data science se od studentů počítá s~výraznějším vhledem do problematiky lineární algebry (spolu s~diskrétní matematikou) a~statistického programování. U~datové analytiky se naopak prioritizuje používání modelových příkladů k~osvojení aplikací analytických technik a~nástrojů (data science pracuje s~spíše s~tvorbou algoritmů, na nichž jsou tyto nástroje postaveny). Druhým výrazným rozdílem je složka týkající se vizualizací, kterou, jak bylo řečeno výše, obsahují obě disciplíny. Nicméně právě datová analytika vnímá vizualizace jako efektivní prostředek ke komunikaci daných informací, narozdíl od data science, jež s~vizualizacemi pracuje spíše jako s~reprezentaci určitých dat.~\parencite[110]{aasheim2015}

Druhý odborný článek zkoumá rozsah dovedností pro vykonávaní těchto disciplín\footnote{Studie zpracovává mimo dvě zmíněné oblasti ještě domény týkající se business analytics a~business intelligence analyst – my ale tyto dvě oblasti sloučíme, protože se jedná o~obsahově obdobné disciplíny, jejichž vydělení není pro charakterizaci datové analytiky až tak relevantní.} na základě analýzy online inzercí pracovních pozic a~jejich požadavků vztahujících se k~daným oblastem. Výsledkem je i~zde porovnání příbuzných a~rozdílných kompetencí a~je velmi zajímavé, že závěry lze bez velkých obtíží vztáhnout k~výsledkům práce popsané výše. Není tedy velkým překvapením, že i~v~této studii vyšel za hlavní rozdíl mezi datovou analýzou a~data science důraz na technickou stránku věci (programování, práce se statistkou atd.), který je primárně součástí data science. Na druhou stranu zde byl u~obou disciplín popsán požadavek na kompetence spojené s~rozhodováním postaveném na datovém základě -- tzn. dovednosti týkající se tvorby efektivní reportu spolu s~analytických myšlením.~\parencite[248]{verma19}

V~rámci porovnání datové analytiky s~business analytics můžeme vidět větší množství vzájemných přesahů v~oblasti měkkých dovedností (komunikace, týmová spolupráce, organizační dovedností atd.), protože doména business analytics je primárně netechnickou disciplínou, u~níž je zapotřebí minimální vhled do problematiky statistických nástrojů.~\parencite[249]{verma19}

Datovou analytiku lze tedy zařadit mezi obě zaměření, protože její kompetence jsou jak technické povahy ve smyslu aplikace základních analytických nástrojů, tak i~charakteru interpretačního, jehož cílem je zpracovaná data vizuálním způsobem komunikovat a~na základě nich tak dojít k~určitému rozhodnutí.
