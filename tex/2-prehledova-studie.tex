\hypertarget{pux159ehledovuxe1-studie-online-kurzux16f}{%
\chapter{Přehledová studie online kurzů}\label{pux159ehledovuxe1-studie-online-kurzux16f}}

\hypertarget{charakteristika-pux159ehledovuxe9-studie}{%
\section{Charakteristika přehledové studie}\label{charakteristika-pux159ehledovuxe9-studie}}

V~této kapitole teoretické části se budeme zabývat již existujícími online kurzy, které se určitým způsobem soustředí na problematiku datové analytiky či alespoň zpracovávají některou z~jejích vybraných kompetencí. Našim cílem je v~této sekci systematicky kategorizovat významná e-learningová řešení a~na základě této klasifikace vytvořit určitou formu přehledové studie\footnote{Je si však zapotřebí uvědomit, že svojí povahou je tato bakalářská práce nejblíže aplikační kvalifikační práci, jejímž cílem bylo vytvořit funkční online kurz, z~tohoto důvodu nelze považovat tuto přehledovou studii za plnohodnotnou výzkumnou studii, tak jak je uváděna v~odborné literatuře~\parencite{mares2013}.}. Každý zpracovaný online (vybírali jsme pouze takové kurzy, s~jejichž obsahem se dá alespoň částečně bezplatně pracovat, a~mají tedy nějakou podobu otevřeného vstupu) kurz sestává ze tří oddělených ale vzájemně se doplňujících částí.

\hypertarget{anotace-kurzu-s-odux16fvodnux11bnuxedm-vuxfdbux11bru}{%
\subsection{Anotace kurzu s~odůvodněním výběru}\label{anotace-kurzu-s-odux16fvodnux11bnuxedm-vuxfdbux11bru}}

V~rámci anotace popisujeme základní cíle kurzu, ve stručnosti charakterizujeme jeho obsahovou stránku a~nastiňujeme strukturu, z~níž se daný kurz skládá. Dále se snažíme zdůvodnit, proč jsme daný kurz vybrali, a~pokusíme se jej začlenit do některého z~konceptů informační či datové gramotností, případně do dalších kompetenčních rámců.

\hypertarget{tabulka-s-parametry}{%
\subsection{Tabulka s~parametry}\label{tabulka-s-parametry}}

Podstatnou částí analýzy u~každého z~online kurzů je tabulka s~předem definovanými parametry, která systematicky uvádí základní informace a~přehledným způsobem znázorňuje charakteristiku daného kurzu. Konkrétní tabulku s~parametry k~určitého kurzu vždy přikládáme do seznamu příloh a~prostřednictvím odkazu propojujeme s~anotací. Níže uvádíme seznam vybraných parametrů:

\begin{itemize}
\tightlist
\item
  název;
\item
  použitá platforma;
\item
  datum vzniku;
\item
  specifikace obsahu kurzu (jaká témata kurz reflektuje);
\item
  časová náročnost;
\item
  způsob práce s~hodnocením a~úkoly;
\item
  rozsah;
\item
  forma ukončení;
\item
  začlenění do kompetenčního rámce;
\item
  použité technologie.
\end{itemize}

\hypertarget{popis-integrovanuxfdch-prvkux16f}{%
\subsection{Popis integrovaných prvků}\label{popis-integrovanuxfdch-prvkux16f}}

Poslední složkou přehledu je popis prvků, které jsme se rozhodli integrovat do našeho praktického řešení. V~této části se nevážeme pouze na jednotlivá témata či použité nástroje, ale všímáme si také zvolených edukačních prostředků. Může jít tedy například o~způsoby průběžného ověřování znalosti a~dovedností, zvolené aktivizační metody za účelem větší interaktivity nebo možnosti získávání zpětné vazby na jednotlivé obsahové moduly. Tyto prvky nám dále poslouží k~návrhu zbývajících částí online kurzu a~případně k~redesignu stávajících komponent.

\hypertarget{online-kurzy}{%
\section{Online kurzy}\label{online-kurzy}}

\hypertarget{introduction-to-data-analytics-for-business}{%
\subsection{Introduction to Data Analytics for Business}\label{introduction-to-data-analytics-for-business}}

\hypertarget{anotace}{%
\subsubsection{Anotace}\label{anotace}}

Kurz \emph{Introduction to Data Analytics for Business} na platformě Coursera je součástí širší specializace pod názvem \emph{Advanced Business Analytics Specialization} a~je zprostředkován pod záštitou University of Colorado Boulder. Tento kurz je úvodní částí zmíněné specializace a~jeho cílem je popsat základní principy datové analytiky v~podnikovém prostředí. Obsah kurzu se velkou mírou soustředí na integraci datové analytiky do procesů určité organizace a~na konkrétních příkladech ukazuje, proč je důležité tuto oblast reflektovat na individuální úrovni ve smyslu zavádění analytických technik do vlastí pracovních mikroprocesů, a~tím zvyšovat celo-organizační efektivitu. Kurz každopádně nezůstává pouze u~teoretických základů, ale v~pozdějších částech pracuje s~využitím konkrétních nástrojů a~představuje základní analytické techniky na úrovni relačních databází a~dotazovacího jazyku SQL.~\parencite{course1}

Struktura kurzu je tvořena čtyřmi na sebe navazujícími sekcemi:

\begin{itemize}
\tightlist
\item
  \emph{Data and Analysis in the Real World};
\item
  \emph{Analytical Tools};
\item
  \emph{Data Extraction Using SQL};
\item
  \emph{Real World Analytical Organizations}.
\end{itemize}

Tento kurz jsme do našeho výběru zahrnuli jednak proto, že se jedná o~jeden z~nejlépe hodnocených kurzů na platformě Coursera zabývající se našim tématem, a~jednak z~toho důvodu, že se nezaměřuje pouze na představení konkrétního analytického nástroje, ale snaží se reflektovat motivaci, která je příčinou pro využívání datové analytiky v~podnikovém prostředí.

Systematickou analýzu kurzu uvádíme v~tabulce s~parametry viz \ref{tab1}.

\hypertarget{integrovanuxe9-prvky}{%
\subsubsection{Integrované prvky}\label{integrovanuxe9-prvky}}

Bla bla bla
