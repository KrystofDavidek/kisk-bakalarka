

\begin{table}[htbp]

\renewcommand\thetable{1}

\caption{Vzdělávací cíle kurzu}\label{tab-cile}

\footnotesize

{

\begin{tabularx}{\linewidth}{p{9cm}c}

\toprule

\multicolumn{2}{c}{\textbf{Teoretická část}}

\tabularnewline
\\

\multicolumn{1}{c}{Vzdělávací cíl} & Kategorie kognitivního procesu 

\tabularnewline
\\
\midrule

1. Student je schopen ilustrovat na fiktivním příkladu důležitost datové analytiky v~modelovém prostředí určité organizace\label{1-cil}

&

Porozumět (dávat příklady)

\\
\midrule

2. Student je schopen rozlišit významy pojmů vztahující se k datové analytice a~k~příbuzným disciplínám\label{2-cil}

&

Porozumět (klasifikovat)

\\

\midrule

3. Student je schopen kategorizovat jednotlivé typy dat a~základní datové formáty používané v~rámci tabulkových dat\label{3-cil}

&

Porozumět (srovnávat)

\\
\toprule

\multicolumn{2}{c}{\textbf{Praktická část}}

\\

\tabularnewline

\multicolumn{1}{c}{Vzdělávací cíl} & Kategorie kognitivního procesu 

\tabularnewline
\\
\midrule

4. Student je schopen vysvětlit, z~jakých částí se skládá proces datové analytiky a~zdůvodnit existenci jednotlivých části\label{4-cil}

&

porozumět (interpretovat)

\\

\midrule

5. Student je schopen použít základní techniky importování dat nehledě na využívaný nástroj\label{5-cil}

&

aplikovat (realizovat)

\\

\midrule

6. Student je schopen identifikovat chyby v∞datech a~dovede tyto problémy řádně opravit\label{6-cil}

&

analyzovat (rozlišovat)

\\

\midrule

7. Student je schopen propojit jednotlivé části dat a~vytvořit na základě nich smysluplné vztahy\label{7-cil}

&

analyzovat (uspořádat)

\\

\midrule

8. Student je schopen vytvořit na základě analyzovaných dat novou informaci, kterou dovede komunikovat prostřednictvím jednoduché vizualizace\label{8-cil}

&

analyzovat (strukturovat)

\\

\midrule

9. Student je schopen vybrat konkrétní analytický nástroj na základě své informační potřeby\label{9-cil}

&

porozumět (srovnávat)

\\

\tabularnewline
\bottomrule
\end{tabularx}

}

\end{table}
