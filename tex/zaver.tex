\hypertarget{zuxe1vux11br}{%
\chapter*{Závěr}\label{zaver}\addcontentsline{toc}{chapter}{Závěr}}

Cílem této bakalářské práce bylo navrhnout a~implementovat online kurz zaměřený na výuku základů datové analytiky. Kurz je určen pro začátečníky, kteří si jeho prostřednictvím osvojí jak teoretické, tak praktické kompetence spojené s~tématem datové analytiky. Tvorbě kurzu předcházela přehledová studie vybraných e-learningů, v~rámci které byly identifikovány dílčí prvky, jež jsme využili v~našem kurzu.

Online kurz se podařilo vytvořit dle zadaných požadavků a~v~aktuální chvíli probíhá uživatelské testování v~rámci interních klientů firmy Digiskills, s~níž jsme návrh kurzu konzultovali. Jelikož získávání zpětné vazby na vytvořený online kurz stále probíhá, nelze v~tuto chvíli diskutovat výsledky testování. Je ale pravděpodobné, že se některé části kurzu mohou na základě získaných odpovědí v~průběhu nejbližší doby mírně změnit.

Na vytvořený online kurz lze dále navázat několika různými způsoby. Jelikož se jedná o~úvodní vstup do problematiky, může být e-learning rozšířen o~pokročilejší témata týkající se složitějších statistických přístupů, které lze aplikovat na větší soubor dat (tedy částečně propojit datovou analytiku s~disciplínou data science). Další cestou může být komplexnější představení konkurenčních analytických nástrojů, s~nimiž se student může setkat napříč odlišnými organizacemi. Třetí možnost je akcentovat témata související s~vizualizací dat a~směřovat studenta spíše do oblasti business analytics, v~rámci které je důležitější efektivnější komunikace informací, nežli provádění náročnějších analýz.

Cíle této práce byly splněny a~její výstupy můžou být rozšířeny v~rámci případného navazujícího aplikovaného výzkumu týkající se uživatelského testování.
