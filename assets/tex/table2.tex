\begin{landscape}

\begin{table}[htbp]

\renewcommand\thetable{2}

\caption{\textit{Intro to Data Analysis}}\label{tab2}

\footnotesize

{

\justifying

\begin{tabularx}{\linewidth}{CCCCC}

\toprule

\textbf{Název} &

\textbf{Použitá platforma} &

\textbf{Datum vzniku} &

\textbf{Časová náročnost} &
 
\textbf{Specifikace obsahu kurzu}

\\

\tabularnewline
\midrule

Intro to Data Analysis

&

Udacity

&

–

&

cca 6 týdnů

&

Problémy řešené datovou analytikou, proces datové analytiky, import a čištění dat, základní analýza a vizualizace dat

\\
\toprule

\textbf{Způsob práce s hodnocením a úkoly} &

\textbf{Forma ukončení} &

\textbf{Použité technologie} &

\textbf{Kompetenční rámec} 

\\

\tabularnewline
\midrule

Bez průběžných úkolů, pouze finální praktický úkol tvořen ze všech oblastí, který je buď vrácen nebo přijat

&

Pro postup do navazujícího kurzu je zapotřebí odevzdat korektně zpracovaný finální úkol

&

Python, NumPy, Pandas, Matplotlib

&

Datová gramotnost – přístup k datům a jejich sběr, prezentace výsledků získaných ze zpracovaných dat, evaluace kvality získaných dat

Datová analytika – proces datové analytiky, základy statistické analýzy, práce s datasety

\\

\tabularnewline
\bottomrule
\end{tabularx}

}

\end{table}

\end{landscape}