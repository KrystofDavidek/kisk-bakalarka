\part{Teoretická část}

\hypertarget{datovuxe1-analytika}{%
\chapter{Datová analytika}\label{datovuxe1-analytika}}

Ústředním tématem této práce je fenomén datové analytiky, jejímž význačným rysem je vágnost přesné definice, protože se na ni v~odlišných prostředích a~kontextech nahlíží různým pohledem dle daných potřeb. Abychom byli schopni provést systematickou přehledovou studii online kurzů zabývajících se touto problematikou a~zároveň při návrhu vlastního e-learningového řešení vycházet z~určitého kompetenčního rámce, se v~této kapitole pokusíme tuto disciplínu zarámovat do konceptu datové gramotnosti. Ve druhé části této sekce se pak zaměříme na konkrétní kompetence, jež jsou součástí datové analytiky, a~z~nichž budeme v~dalších částech vycházet.

\hypertarget{datovuxe1-gramotnost}{%
\section{Datová gramotnost}\label{datovuxe1-gramotnost}}

Při práci s~konceptem datové gramotnosti se můžeme setkat s~různými pojetími, které se buď vzájemně překrývají, zastupují či doplňují. Dle některých autorů lze datovou gramotnost zahrnout pod gramotnost informační, protože právě ta může ve své obecné definici zahrnovat i~práci s~daty s~jakožto speciálním druhem informace.~\parencite[126]{calzada13}

Například dle standardu instituce ACRL\footnote{The Association of College and Research Libraries – http://www.ala.org/acrl/} je informační gramotnost definována jako souhrn schopností a~dovedností jedince, které slouží k~identifikaci své informační potřeby a~jejímu následnému uspokojení protřednictvím lokalizace, evaluace (zhodnocení) a~efektivního využití dané informace.~\parencite[2]{acrl06} Podobným způsobem definuje informační gramotnost i~TDKIV\footnote{https://tdkiv.nkp.cz/}: \uv{Schopnost jedince identifikovat informační potřebu, vyhledat informace, zhodnotit je, zpracovat a~efektivně využít.}~\parencite{tdkiv03}

Jiní autoři pracují s~významem datové gramotnosti ve spojitosti se statistickou gramotností\footnote{Pod statistickou gramotností si lze představit dovednosti spojené se čtením, interpretací a~tvorbou statistických dat jako jsou grafy a~tabulky, ale i~číselná data~\parencite[7]{schield05} – my tyto specifické dovednosti často techničtější povahy chápeme jako součást datové gramotnosti}, resp. uvažují informační gramotnost jako prerekvizitu k~datové gramotností, jelikož práce s~daty vyžaduje kritický pohled na informace -- jejích správně čtení, hodnocení a~interpretaci. Bez těchto složek si lze pak těžko představit datovou gramotnost jakožto proces skládající se z~těchto hlavních částí:~\parencite[8]{schield05}

\begin{itemize}
\tightlist
\item
  přístup k~datům a~jejich sběr;
\item
  posouzení kvality dat;
\item
  manipulace s~daty a~jejich zpracování;
\item
  sumarizace a~zhodnocení dat;
\item
  prezentace dat.
\end{itemize}

Prováznost informační a~datové gramotnosti můžeme dále spatřovat ve standardu informační gramotnosti \emph{The seven pillars of information literacy}, který inherentně předpokládá, že právě data jsou s~informacemi vzájemně spjaty, a~proto lze s~nimi pracovat na principiálně podobné, ne-li stejné úrovni.~\parencite[126]{calzada13} Tento standard nabízí sedm pilířů (oblastí), které je nutné v~rámci práce s~informacemi/daty cyklicky provádět.~\parencite{sconul11} Předchozí výčet dovedností a~schopností bychom tak mohli modifikovat následovně:

\begin{itemize}
\tightlist
\item
  identifikace datové potřeby;
\item
  zhodnocení aktuálních znalostí a~případné zjištění nedostatků;
\item
  vytvoření strategie pro vyhledání potřebných dat;
\item
  schopnost přistoupit k~datům a~získat je;
\item
  evaluace kvality získaných dat;
\item
  etická manipulace s~daty a~jejich zpracování;
\item
  prezentace výsledků získaných ze zpracovaných dat.
\end{itemize}
