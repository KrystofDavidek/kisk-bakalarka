\hypertarget{uxfavod}{%
\chapter*{Úvod}\label{uvod}\addcontentsline{toc}{chapter}{Úvod}}

Sousloví datová analytika bývá v~dnešní digitální době často skloňovaným pojmem, nicméně už ne každý je schopen jasně vysvětlit, o~čem tato disciplína je a~k~čemu se může hodit pochopit její základy. I~přes to, že v~českém prostředí existují partikulární kurzy či dílčí vzdělávací materiály, které se zabývají určitou částí datové analytiky, neexistuje velké množství ucelených online kurzů, jež tohle téma přehledně uvádí jak z~teoretického, tak z~praktického hlediska.

V~této bakalářské práci se proto zabýváme analýzou již existujících online kurzů zaměřených na výuku datové analytiky a~na základě těchto poznatků navrhujeme nový online kurz, který tohle téma představuje od úplných začátků. Hlavním cílem této práce je tedy navrhnout a~implementovat online kurz zaměřený na výuku základů datové analytiky. Sekundárním cílem je pak systematicky popsat významné zahraniční e-learningy zabývající se obdobným tématem a~rozvést jejich funkční či nefunkční prvky.

Práce se skládá ze dvou částí -- teoretické a~praktické. V~teoretické části je nejprve představen fenomén datové analytiky v~kontextu datové a~digitální gramotnosti. V~těchto úvodních kapitolách rovněž uvádíme některé přístupy k~zachycení schopností a~dovedností datového analytika a~vytváříme vlastní kompetenční rámec, který nám dále slouží při tvorbě vzdělávacích cílů kurzu.

Hlavní součást teoretické části je pak přehledová studie, jejímž cílem je zmapovat již existující, primárně zahraniční online kurzy zabývající se výukou datové analytiky. Výsledky z~této analýzy využíváme v~návrhu vlastního řešení.

V~praktické části práce, představujme vytvořený online kurzu. Text dělíme do dvou hlavních částí, které vždy reflektují odlišnou rovinu návrhu a~implementace kurzu. Nejprve představujeme koncept online kurzu -- popisujeme hlavní motivace, vytyčení vzdělávacích cílů s~ohledem na revidovanou Bloomovu taxonomii vzdělávacích cílů a~začleňujeme kurz do některého z~kompetenčních rámců.

Poslední část se týká samotného obsahu -- popisuje průchod kurzem, vysvětluje zvolené edukační komponenty a~představuje obsahy jednotlivých vzdělávacích modulů.

Výsledný online kurz vznikl ve spolupráci s~firmou Digiskills, která se zabývá digitálním vzděláváním převážně v~podnikovém prostředí.
