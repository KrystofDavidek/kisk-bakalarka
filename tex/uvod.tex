\hypertarget{uxfavod}{%
\chapter*{Úvod}\label{uvod}\addcontentsline{toc}{chapter}{Úvod}}

Návrh a~implementace online kurzu pro výuku datové analytiky Design and
implementation of an online course for teaching data analytics

\chapter*{Anotace}

Předkládaná bakalářská práce si klade za cíl navrhnout a~implementovat
online vzdělávací kurz, který se zaměřuje na výuku základů datové
analytiky. Práce se dělí do dvou hlavních částí -- teoretické
a~praktické. V~první částí je představen koncept datové analytiky spolu
s~přehledovou studií již existujících, primárně zahraničních online kurzů
zaměřených na související téma. Jejím cílem je zmapovat aktuální stav
online kurzů vzniklých primárně v~zahraničním prostředí
a~prostřednictvím předem jasně definovaných parametrů tyto kurzy
klasifikovat. Praktická část se následně skládá z~popisu vlastního
řešení, které tak částečně vychází z~výsledků provedené přehledové
studie. Implementace kurzu probíhala za spolupráce s~firmou Digiskills,
pro jejichž potřeby a~převážně jejich klienty byl tento kurz navržen.
Pro zpracování online kurzu byly využity interní webové šablony firmy,
všechny zbylé vzdělávací materiály byly vytvořený exkluzivně pro kurz
jako takový.

\begin{center}\rule{0.5\linewidth}{0.5pt}\end{center}

\begin{longtable}[]{@{}ll@{}}
\toprule
Hlavička 1 & Hlavička 2 \\
\midrule
\endhead
Element 1 & Element 2 \\
\bottomrule
\end{longtable}

Úvod

Teoretická část

Datová analytika -- co to je? Datová gramotnost Přístupy k~datové
analytice Přehledová studie Charakteristika přehledové studie --
motivace 1. Motivace za výběrem daného kurzu 2. Tabulka s~předem
definovanými parametry: a. Název kurzu b. Použitá platforma c.~Datum
vzniku kurzu d.~Specifikace obsahu kurzu e. Časová náročnost f.~Způsob
práce s~úkoly g. Rozsah kurzu h. Forma ukončení i. Začlenění do modelu
informační/datové gramotnosti j. Závislost na konkrétním nástroji 3.
Anotace kurzu spolu s~případným popisem prvků, které jsme integrovali do
našeho praktického řešení

Praktická část

Koncept Motivace Cíl Začlenění do modelu IG Struktura kurzu Průchod
Použité komponenty Obsah modulů Popis modulu + Bloomova taxonomie +
případně (E)UR + formativní charakter testů Výsledky uživatelského
testování???

Závěr
