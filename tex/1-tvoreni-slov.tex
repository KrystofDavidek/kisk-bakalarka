\part{Teoretická část}

\hypertarget{slovotvorba}{%
\chapter{Slovotvorba}\label{slovotvorba}}

V~této kapitole si představíme slovotvorbu tak, jak ji vnímá dnešní
lingvistika, ve stručnosti si popíšeme hlavní přístupy k~této
problematice v~českém prostředí a~nastíníme vztah pojmu derivační
morfologie k~této disciplíně.

Za slovotvorbu lze v~lingvistice v~nejobecnějším pojetí považovat nauku
o~tvoření slov, jde tedy o~takovou vědní disciplínu, která zkoumá
a~popisuje procesy, které doprovází vznik nových pojmenování v~daném
jazyce.

V~užším pojetí pod tvořením slov myslíme právě ty procesy, jež pod sebou
zahrnují různé způsoby, postupy a~prostředky, díky nimž dochází
k~produkci nových slov (tyto slovotvorné postupy jsou často
reprodukovány). Důležitou součástí slovotvorby jsou pak výsledné stavy
těchto procesů, které označujeme jako slovotvornou stavbu.
\parencite[92]{dokulil00}

Vznik (geneze) nového slova je tak odrazem určité slovotvorné stavby --
to znamená, že slovotvorná stavba ukazuje, jak se z~dnešního pohledu
určité slovo jeví být z~jiného slova utvořeno (například u~slova
\emph{ne-dobrý} lze vyčíst, že je pomocí prefixu \emph{ne} odvozeno od
adjektiva \emph{dobrý}).~\parencite[92--93]{dokulil00} Ne vždy ale může
být u~odvozovaných slov zřejmé, které slovo je tvarem základovým, tato
problematika je i~s~příklady rozebrána v~podkapitole
\ref{slovotvornuxe9-vztahy}.

\hypertarget{pux159uxedstupy-a-teorie}{%
\section{Přístupy a~teorie}\label{pux159uxedstupy-a-teorie}}

Jako většina lingvistických disciplín, prošla si i~tato oblast svým
vlastním vývojem, jejž dobře mapuje kapitola \emph{Slovotvorba}
v~publikaci Zdenky Rousínové -- \emph{Dějiny českého jazykovědného
myšlení}. Vzhledem k~účelům této práce ponecháváme historický exkurz
stranou a~zaměřujeme se na nejvýznamnější synchronní přístupy v~rámci
tvoření slov ve 20. a~21. století.

Za jedno z~prvních významných děl lze považovat \emph{Mluvnici spisovné
češtiny} od Františka Trávníčka (1. vydání vyšlo v~roce 1948), která
jako jedna z~prvních gramatik synchronně popisuje problematiku
slovotvorby. Zde je nutno podotknout, že Trávníček tvoření nových slov
úzce spojoval s~lexikologií, a~nevyčlenil jej tak jako samostatnou
disciplínu.~\parencite[263]{rousinova07}

Další výrazný příspěvek do teorie slovotvorby představuje publikace
Vladimíra Šmilauera \emph{Novočeské tvoření slov}, která byla napsána
v~letech 1937--1938, nicméně z~geopolitických důvodů dílo vyšlo až v~roce
1971. Šmilauer se zde primárně zaměřil na popis procesu, kterým si
procházejí mluvčí, chtějí-li vytvářet nové pojmenování pro nějaký jev.
\parencite[265]{rousinova07}

Přelomovým dílem, které proměnilo českou slovotvorbu do dnešních podob,
je publikace Miloše Dokulila \emph{Teorie odvozování} (vyšla v~roce
1962). Tato práce byla první částí kolektivního díla \emph{Tvoření slov
v~češtině}, na níž se podílel kolektiv pracovníků Ústavu pro jazyk český
Československé akademie věd. Cílem tohoto textu bylo podat komplexní
metodologický základ, jenž měl být svoji univerzálností platný i~na
další slovanské jazyky.~\parencite[267]{rousinova07} Tato práce dala za
vznik všeobecně přijímané onomaziologické teorii slovotvorby (viz
kapitola \ref{onomaziologickuxe1-teorie-slovotvorby}), na níž bylo
navázáno dalšími
pracemi\footnote{Teorie se dočkala dopracování ostatních slovních druhů v~následujících publikacích -- Mluvnice češtiny (1986), Příruční mluvnice češtiny (1995) a~Čeština -- řeč a~jazyk (1996). Dokulilovo dílo bylo dále zpracováno v~jednotlivých heslech v~rámci Encyklopedického slovníku češtiny.~\parencite[272]{rousinova07}},
které teorii dále rozvíjely, aniž by ztratila na své platnosti.
\parencite[272]{rousinova07} I~díky tomu se do teď tato teorie považuje
za nejúplnější analýzu slovotvorné struktury češtiny.
\parencite[273]{zikova07}

I~přes všeobecné přijetí Dokulilovy teorie se na počátku 21. století
objevují alternativní přístupy. Za jeden z~nejvýznamnějších se považují
valenční analýzy struktury slov, které zkoumají chování valenčních rámců
-- zjednodušeně tedy to, jak se mění syntaktické vlastnosti v~rámci
odvozených slov. Z~této teorie vychází například Jarmila Panevová nebo
Petr Karlík, kteří se zabývají valencí deverbálních jmen (substantiv,
která vznikla odvozením z~verb). Jedna z~posledních významných
alternativních teorií, která nebyla na našem území příliš reflektována,
je generativní syntax, která se projevila spíše v~anglosaských zemí
a~jejímž zakladatelem je Noam Chomsky.~\parencite[274--275]{zikova07}

Za jedno z~nejnovějších děl pojednávajících o~české slovotvorbě lze
považovat \emph{Velkou akademickou gramatiku spisovné češtiny. I.
Morfologie: Druhy slov / Tvoření slov. Část 1., Část 2.} (1. vydání
vyšlo v~roce 2018). Tato deskriptivní gramatika, napsaná Františkem
Štíchou, sice částečně vychází z~tradiční Dokulilovy teorie, nicméně se
nebrání popisovat vybrané slovotvorné jevy podle vlastního pojetí.
Příkladem může být problematika názvů osob pojmenovaných vzhledem
k~charakteristické vlastnosti odvozených z~adjektiv. Dokulil tuto
kategorii označuje za \emph{nositele vlastností}, nicméně Štícha tuto
třídu explicitně nevyčleňuje a~vztahuje tento termín obecně na kategorii
názvů osob odvozených z~adjektiv. Důležitou charakteristikou této
gramatiky je taktéž práce s~autentickými texty z~jazykových databází
Českého národního korpusu.~\parencite{sticha18}

\hypertarget{derivaux10dnuxed-morfologie}{%
\section{Derivační morfologie}\label{derivaux10dnuxed-morfologie}}

Pojetí slovotvorby se může napříč různými přístupy značně odlišovat,
v~anglosaských zemích na rozdíl od slovanské tradice existuje označení
derivační morfologie (\emph{derivational morphology}), která spolu
s~flektivní morfologií tvoří morfologii jako celek (tento přístup je
v~českém jazyce taktéž reflektován, nicméně nepřevládá).
\parencite{lieber14}

V~českém prostředí se tradičně slovotvorba od morfologie odděluje. Na
úrovni morfologie se tradičně pracuje s~morfémy jako s~prostředky mající
primárně gramatickou funkci (flektivní morfologie v~anglosaském pojetí),
a~nikoliv jako s~prostředky slovotvornými. Pojem derivační morfologie je
tedy srovnatelný s~pojmem derivace jakožto s~jedním ze tří slovotvorných
způsobů české slovotvorby.
