\hypertarget{pux159ehledovuxe1-studie-online-kurzux16f}{%
\chapter{Přehledová studie online kurzů}\label{pux159ehledovuxe1-studie-online-kurzux16f}}

\hypertarget{charakteristika-pux159ehledovuxe9-studie}{%
\section{Charakteristika přehledové studie}\label{charakteristika-pux159ehledovuxe9-studie}}

V~této kapitole teoretické části se budeme zabývat již existujícími online kurzy, které se určitým způsobem soustředí na problematiku datové analytiky či alespoň zpracovávají některou z~jejích vybraných kompetencí. Našim cílem je v~této sekci systematicky kategorizovat významná e-learningová řešení a~na základě této klasifikace vytvořit určitou formu přehledové studie\footnote{Je si však zapotřebí uvědomit, že svojí povahou je tato bakalářská práce nejblíže aplikační kvalifikační práci, jejímž cílem bylo vytvořit funkční online kurz, z~tohoto důvodu nelze považovat tuto přehledovou studii za plnohodnotnou výzkumnou studii, tak jak je uváděna v~odborné literatuře~\parencite{mares2013}.}. Každý zpracovaný online kurz sestává ze tří oddělených ale vzájemně se doplňujících částí.

\hypertarget{anotace-kurzu-s-odux16fvodnux11bnuxedm-vuxfdbux11bru}{%
\subsection{Anotace kurzu s~odůvodněním výběru}\label{anotace-kurzu-s-odux16fvodnux11bnuxedm-vuxfdbux11bru}}

V~rámci anotace popisujeme základní cíle spolu s~obsahem kurzu a~ve stručnosti charakterizujeme platformu, na níž je daný kurz umístěn. Dále se snažíme zdůvodnit, proč jsme daný kurz vybrali a~pokusíme se jej začlenit do některých z~konceptů informační či datové gramotností, případně do dalších kompetenčních rámců.

\hypertarget{tabulka-s-parametry}{%
\subsection{Tabulka s~parametry}\label{tabulka-s-parametry}}

Hlavní částí analýzy u~každého z~online kurzů je tabulka s~předem definovanými parametry, která systematicky uvádí základní informace a~přehledným způsobem znázorňuje charakteristiku daného kurzu. Níže uvádíme seznam vybraných parametrů:

\begin{itemize}
\tightlist
\item
  název;
\item
  použitá platforma;
\item
  datum vzniku;
\item
  specifikace obsahu kurzu (jaká témata kurz reflektuje);
\item
  časová náročnost;
\item
  způsob práce s~hodnocením a~úkoly;
\item
  rozsah;
\item
  forma ukončení;
\item
  začlenění do kompetenčního rámce;
\item
  použité technologie.
\end{itemize}

\hypertarget{popis-integrovanuxfdch-prvkux16f}{%
\subsection{Popis integrovaných prvků}\label{popis-integrovanuxfdch-prvkux16f}}

Poslední složkou přehledu je popis prvků, které jsme se rozhodli integrovat do našeho praktického řešení. V~této části se nevážeme pouze na jednotlivá témata či použité nástrojů, ale všímáme si také zvolených edukačních prostředků. Může jít tedy například o~způsoby průběžného ověřování znalosti a~dovedností, zvolené aktivizační metody za účelem větší interaktivity nebo možnosti získávání zpětné vazby na jednotlivé obsahové moduly. Tyto prvky nám dále poslouží k~návrhu zbývajících částí online kurzu a~případě k~redesignu stávajících komponent.

https://www.coursera.org/learn/data-analytics-business?authMode=signup\#syllabus

https://classroom.udacity.com/courses/ud170/lessons/5430778793/concepts/53994889400923

https://www.czechitas.cz/cs/kalendar-akci/akce/9032

https://thedataliteracyproject.org/learn

https://learning.edx.org/course/course-v1:DavidsonX+DavidsonX.D006+1T2021/home

https://learning.edx.org/course/course-v1:DavidsonX+DavidsonX.D005+1T2020/home

https://www.coursera.org/learn/introduction-to-data-analytics?action=enroll

https://www.coursera.org/learn/excel-basics-data-analysis-ibm?specialization=ibm-data-analyst\#syllabus

https://www.coursera.org/learn/data-visualization-dashboards-excel-cognos?specialization=ibm-data-analyst\#syllabus

https://www.futurelearn.com/subjects/business-and-management-courses/big-data-analytics/data-analytics
